\chapter{Técnicas de integração}

\section{Integrais trigonométricas}

\subsection{\(sen^n(x).cos^m(x)\)}

\subsubsection{\(m\) é impar}

\begin{flalign}
    \int sen(x)^n.cos^m(x)\; dx &&
\end{flalign}
se \(m\) é impar \(\implies\) \(m = 2k + 1\)
\begin{flalign}
    \int sen(x)^n.cos^{2k + 1}(x)\; dx = &&
\end{flalign}
\begin{flalign}
    \int sen(x)^n.(cos^{2}(x))^kcos^{1}(x)\; dx = &&
\end{flalign}
Utilizando a identidade \( cos^2(x) = 1 - sen^2(x)\),
\begin{flalign}
    \int sen(x)^n.(1 - sen^2(x))^k cos^{1}(x)\; dx &&
\end{flalign}
Daí, tomando \( u = sen(x)\), temos \( du = cos(x)dx\) e segue que:
\begin{flalign}
    \int u^n.(1 - u^ 2)^k u\; du &&
\end{flalign}

\subsubsection{\(n\) é impar}

O processo é análogo, 

\begin{enumerate}
    \item Guarde um fator seno
    \item Use a identidade \( sen^2(x) = 1 - cos^2(x) \)
    \item Substitua \( u = cos(x) \)
    \item Resolva a integral
\end{enumerate}

\subsubsection{\(n\) e \(m\) são pares}

Utilizamos as identidades:
\begin{gather}
    sen^2(x) = \frac{1}{2} ( 1 - cos(2x)) \\
    cos^2(x) = \frac{1}{2} ( 1 - sen(2x)) \\
    cos(x)sen(x) = \frac{1}{2} sen(2x)
\end{gather}

\subsection{\(sen(mx).cos(nx)\)}

\begin{flalign}
    \int sen(mx).cos(nx)\; dx &&
\end{flalign}
Utilizamos a seguinte Identidade:

\begin{equation}
    sen(a)cos(b) = 
    \frac{1}{2} 
    [
    sen(a - b)
    +
    sen(a + b
    ]
\end{equation}

\begin{flalign}
    \int sen(mx).sen(nx)\; dx &&
\end{flalign}
Utilizamos a seguinte Identidade:

\begin{equation}
    sen(a)sen(b) = 
    \frac{1}{2} 
    [
    cos(a - b)
    -
    cos(a + b
    ]
\end{equation}

\begin{flalign}
    \int cos(mx).cos(nx)\; dx &&
\end{flalign}
Utilizamos a seguinte Identidade:

\begin{equation}
    sen(a)sen(b) = 
    \frac{1}{2} 
    [
    cos(a - b)
    +
    cos(a + b
    ]
\end{equation}

\subsection{\(tan^n(x).sec^m(x)\)}

\subsubsection{\(n\) é par}
\begin{enumerate}
    \item Guarde um fator \(sec^2(x)\)
    \item use que \(sec^2(x) = 1 + tan^2(x)\)
    \item substitua u = tan(x)
\end{enumerate}

\subsubsection{\(n\) é impar}

\begin{enumerate}
    \item Guarde um fator \(sec(x)tan(x)\)
    \item use que \(tan^2(x) = sec^2(x) - 1\)
    \item substitua u = sec(x)
\end{enumerate}

\subsubsection{\(n\) par e \(m\) ímpar}

\begin{enumerate}
    \item Expresse tudo em termos da secante
    \item Resolva as potencias da secante integrando por partes
\end{enumerate}

utilizando as integrais primitivas:
\begin{flalign}
    \int tan(x)\; dx = ln(|sec(x)|) + c
\end{flalign}
\begin{flalign}
    \int sec(x)\; dx = ln(|sec(x)|) + tan(x) + c
\end{flalign}

\section{Substituição trigonométrica}

\subsection{\(\sqrt{a^2 - x^2}\)}
Substitua:

\begin{equation}
    x = a.sen(\theta)
\end{equation}
com o triangulo representado por:
\begin{equation}
    a^2 = (\sqrt{a^2 - x^2})^2 + x^2
\end{equation}

\subsection{\(\sqrt{a^2 - x^2}\)}
Substitua:

\begin{equation}
    x = a.tan(\theta)
\end{equation}
com o triangulo representado por:
\begin{equation}
    (\sqrt{a^2 + x^2})^2 = a^2 + x^2
\end{equation}

\subsection{\(\sqrt{x^2 - a^2}\)}
Substituía:

\begin{equation}
    x = a.sec(\theta)
\end{equation}
com o triangulo representado por:
\begin{equation}
    x^2 = (\sqrt{x^2 - a^2})^2 + a^2 
\end{equation}


\section{Funções racionais}

Temos a integral 

\begin{equation}
    \int \frac{p(x)}{q(x)}\; dx
\end{equation}

Se \(grau(p) > grau(q)\) então efetuamos a divisão o que nos da temos:

\begin{equation}
    f(x) = \frac{p(x)}{q(x)} = \frac{r(x)}{q(x)} + s(x) 
\end{equation}
Onde \(grau(r) > grau(q)\). E nossa integral se transforma em:
\begin{equation}
    \int f(x) = \int \left( \frac{r(x)}{q(x)} + s(x)\; \right) dx
\end{equation}
E daí temos que:
\begin{equation}
    \int f(x) = \int \frac{r(x)}{q(x)}\; dx + \int s(x)\; dx 
\end{equation}
fatorando ambos os polinômios de \(q(x)\) temos que:

\begin{equation}
    \textstyle
    f(x) = \frac{
        r(x)
    }{
        (a_1x + b_1)^{n_1}
        \cdot ... \cdot 
        (a_kx + b_k)^{n_k}
        \cdot
        (d_1x^2 + e_1x + f_1)^{m_1}
        \cdot ... \cdot 
        (d_lx^2 + e_lx + f_l)^{m_l}
    }
\end{equation}
Onde cada \(a_ix + b_i, i = 1...k\), são equações lineares, e as equações \(d_jx^2 + e_jx + f_j, j = 1...l\), são equações de segundo grau irredutíveis \(( \Delta < 0 )\).
Queremos cada componente da fatoração como:
\\
\\
se a \(c_n(x)\) é a \textit{n-ésima} componente da fatoração de \(q(x)\) e tivermos \(c_n(x) = (a_ix + b_i)^k\) fatoramos como:
\begin{equation*}
    c_n(x) = 
        \frac{A_1}{(a_1x + b_1)^1} +
        \frac{A_2}{(a_2x + b_2)^2} + ... + 
        \frac{A_n}{(a_nx + b_n)^n}
\end{equation*}
que denotamos como:
\begin{equation}
    c_n(x) = \sum_{ i = 1}^{n} \frac{A_i}{(a_1x + b_1)^i}
\end{equation}
se a \(d_n(x)\) é a \textit{n-ésima} componente da fatoração de \(q(x)\) e tivermos \(d_n(x) = (a_jx^2 + b_jx + c_j)^l\) fatoramos como:
\begin{equation}
    d_n(x) = 
        \frac{B_1x + C_1}{(c_1x^2 + d_1x + e_1)^1} +
        \frac{B_2x + C_2}{(c_2x^2 + d_2x + e_2)^2} + ... + 
        \frac{B_nx + C_m}{(c_mx^2 + d_mx + e_m)^m}
\end{equation}
que denotamos como:
\begin{equation}
    d_n(x) = \sum_{j = 1}^{m} \frac{B_jx + C_j}{(c_jx^2 + d_jx + e_j)^j}
\end{equation}
Temos uma fatoração de \(f(x)\) de tal forma:
\begin{equation}
    \frac{p(x)}{q(x)} =
    \sum_{u = 1}^{v} 
        c_u(x) + 
    \sum_{w = 1}^{z} 
        d_w(x)
\end{equation}
teremos os coeficientes \(A_p, B_q, C_r\) resolvendo o sistema linear que decorre de:
\begin{equation}
    q(x) = 
        p(x)
            \left(
                \sum_{u = 1}^{v} 
            c_u(x) + 
                \sum_{w = 1}^{z} 
            d_w(x)
            \right)
\end{equation}
Descobertos os coeficientes das equações, teremos que:
\begin{equation}
    f(x) = 
        s(x) + \frac{p(x)}{q(x)}
\end{equation}
\begin{equation}
    f(x) = 
        s(x) + 
        \sum_{u = 1}^{v} 
            c_u(x) + 
        \sum_{w = 1}^{z} 
            d_w(x)
\end{equation}
o que nos da finalmente que
\begin{equation}
    \int f(x) dx = 
        \int s(x)dx + 
        \int 
            \left(
                \sum_{u = 1}^{v} c_u(x)
            \right)
        dx
        + 
        \int 
            \left(
                \sum_{w = 1}^{z} d_w(x)
            \right)
        dx
\end{equation}
e como sabemos integral o polinômio \(s(x)\) e cada polinômio \(c_u\) e \(d_w\), temos uma integral simplificada.
\section{Substituição racionalizante}

Se tivermos:
\begin{equation}
    \int \sqrt[\leftroot{-3}\uproot{3}n]{g(x)}\; dx
\end{equation}
Substituímos \( u = \sqrt[\leftroot{-3}\uproot{3}n]{g(x)} \)
\\
Se tivermos \(f(x)\) um quociente entre 2 polinomios com potencias fracionarias de x, então fazemos a substituição 
\begin{equation}
    x = t^k
\end{equation} 
Onde \(k\) é denominador comum entre as frações potencia de \(x\)