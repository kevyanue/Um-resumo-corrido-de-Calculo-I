\chapter{Limites}

\section{Regras}

Seja \(C\) uma constante qualquer e suponha que 
\noindent
\begin{equation*}
    \exists L_f ; \lim_{x \to a} f(x) = L_f
    \quad
    e
    \quad
    \exists L_g ; \lim_{x \to a} f(x) = L_g
\end{equation*}
    
\begin{flalign}
    \lim_{ x \to a } 
        \left[
            f(x) \pm g(x)
        \right]
    = 
        \left[
            \lim_{x \to a} f(x)
        \right]
            \pm 
        \left[
            \lim_{x \to a} f(x)]
        \right]
    = 
        L_f + L_g &&
\end{flalign}

\begin{flalign}
    \lim_{ x \to a } 
        \left[
            C.f(x)
        \right]
    = 
        C . \left[
                \lim_{x \to a} f(x)
            \right]
    = 
        C.L_f &&
\end{flalign}

\begin{flalign}
    \lim_{ x \to a } 
        \left[
            f(x) . g(x)
        \right]
    = 
        \left[
            \lim_{x \to a} f(x)
        \right]
        .
        \left[
            \lim_{x \to a} g(x)
        \right]
    = 
        L_f . L_g &&
\end{flalign}

\begin{flalign}
    \lim_{ x \to a } 
        \left[
            \frac{f(x)}{g(x)} 
        \right]
    = 
        \left[
            \frac{
                \lim_{x \to a} f(x)
            }{
                \lim_{x \to a} g(x)
            }    
        \right]
    = 
        \left[
            \frac{
                L_f 
            }{
                L_g
            }
        \right]        
    ,
        \lim_{x \to a} g(x) \neq 0 &&
\end{flalign}

\begin{flalign}
    \lim_{ x \to a } 
        \left[
            f(x)
        \right]^n
    = 
        \left[
            \lim_{x \to a} f(x)
        \right]^n
    = 
        [L_f]^n &&
\end{flalign}

\subsection{Limite da composta}

Sejam \(f\) e \(g\) duas funções tais que
\begin{equation*}
    \lim_{x \to a} g(x) = L_g 
    \quad e \quad
    \lim_{x \to a} f(x) = L_f
\end{equation*}
Se, para \(x\) próximos de a e \(x \neq a\) temos
\begin{equation}
    g(x) \in Df \quad e \quad g(x) \neq L_g
    \implies \lim_{x \to a} (f \circ g)(x) = L_f 
\end{equation}

\subsection{Limite da composta (feat. continuidade)}
Se
\begin{enumerate}
    \item \(\lim_{x \to a} g(x) = L_g\)
    \item \(f\) \text{contínua em} \(L_g\)
    \item \(g(x) \in Df\) para \(x\) próximos de \(a\) e \(x \neq a\)
\end{enumerate}
Então
\begin{equation}
    \lim_{x \to a} f(g(x)) 
    = 
    f \left(
        \;
        \lim_{x \to a} g(x)  
        \;
      \right)
    = f (L_g)  
\end{equation}
a função \(f\) ``saí pra fora'' do limite
\subsection{Mudança de variável}
Considerando o limite:
\begin{equation*}
    \lim_{x \to a} f(g(x))
\end{equation*}
Se definirmos \(u = g(x)\),
temos que
\begin{equation}\label{md}
    x \to a,\, u = g(x) \to b
\end{equation}
O que nos da que
\begin{equation*}
    \lim_{x \to a} f(g(x)) 
    \stackrel{\ref{md}}{=}
    \lim_{u \to b} f(u) 
\end{equation*}
Ou seja, nosso limite dado em função de \(x\) está agora em função de \(u\)
\section{Limites fundamentais}

\begin{flalign}
    \lim_{x \to 0} \frac{\sin{x}}{x} = 1 &&
\end{flalign}

\begin{flalign}
    \lim_{x \to \pm \infty}\left(1+\frac{1}{x}\right)^{x} = e &&
\end{flalign}

\begin{flalign}
    \lim_{x \to 0}\frac{a^{x}-1}{x} = \ln{a} &&
\end{flalign}

\begin{flalign}
    \lim_{x \to 0}\frac{a^{x}-1}{x}=\ln a &&
\end{flalign}

\begin{flalign}
    \lim_{x \to 0^+}\frac{1}{x} = +\infty &&
\end{flalign}

\begin{flalign}
    \lim_{x \to 0^-}\frac{1}{x} = -\infty &&
\end{flalign}

\begin{flalign}
    \lim_{x \to \pm \infty}\frac{1}{x} = 0 &&
\end{flalign}


\subsection{Outros limites}

\begin{flalign}
    \lim_{x\to \pm \infty}\left(1+\frac{a}{x}\right)^{bx}= e^{ab} &&
\end{flalign}

\begin{flalign}
    \lim_{x\to \pm \infty} a^{\frac{1}{x}} = 1 &&
\end{flalign}

\begin{flalign}
    \lim_{x\to \pm \infty} \frac{\ln{n}}{n} = 0 &&
\end{flalign}

\begin{flalign}
    \lim_{x \to 0} \frac{1 - \cos{x}}{x} = 0 &&
\end{flalign}

\section{Regras do infinito}
Aqui considere:
\begin{enumerate}
    \item L é um numero qualquer \textbf{Diferente de 0}
    \item \(L^+\) é um numero qualquer \textbf{estritamente} positivo
    \item \(L^-\) é um numero qualquer \textbf{estritamente} negativo
    \item \(\pm \infty\) como algum limite de alguma função \(f\)
\end{enumerate}
Temos que:
\begin{flalign}
    \infty + \infty = \infty &&
\end{flalign}
\begin{flalign}
    \infty . \infty = \infty &&
\end{flalign}
\begin{flalign}
    L + \infty = \infty &&
\end{flalign}
\begin{flalign}
    \infty - L = \infty &&
\end{flalign}
\begin{flalign}
    L^+ . \infty = \infty &&
\end{flalign}
\begin{flalign}
    L^- . \infty = -\infty &&
\end{flalign}
\begin{flalign}
    L^+ . -\infty = -\infty &&
\end{flalign}
\begin{flalign}
    L^- . -\infty = +\infty &&
\end{flalign}
\begin{flalign}
    -\infty - \infty = -\infty &&
\end{flalign}
\begin{flalign}
    -\infty . (- \infty) = +\infty &&
\end{flalign}

\subsection{Indeterminações com infinitos}

Assuma o mesmo que a seção anterior.
As seguintes expressões são indet.:

\begin{flalign}
    \infty - \infty &&
\end{flalign}
\begin{flalign}
    0 . \infty &&
\end{flalign}
\begin{flalign}
    \frac{\pm\infty}{\pm\infty} &&
\end{flalign}

\begin{flalign}
    \frac{0}{0} &&
\end{flalign}

\begin{flalign}
    1^{\infty} &&
\end{flalign}
\begin{flalign}
    \infty^{0} &&
\end{flalign}
\begin{flalign}
    0^{0} &&
\end{flalign}

\section{Teorema que antecede o confronto}
se temos que 
\begin{equation*}
    f(x) \leq g(x)
\end{equation*}
então
\begin{equation}
    \lim_{x \to a} f(x) 
    \leq 
    \lim_{x \to a} g(x)
\end{equation}
(se os limites existirem)
\section{Teorema do confronto}
sejam \(f\), \(g\) e \(h\) funções. Se temos que:
\begin{enumerate}
    \item \[f(x) \leq g(x) \leq h(x)\]
    \item \[\lim_{x \to a} f(x) = L = \lim_{x \to a} h(x)\] 
\end{enumerate}
Temos que
\begin{equation}
    \lim_{x \to a} f(x)
    \leq 
    \lim_{x \to a} g(x) 
    \leq 
    \lim_{x \to a} h(x) 
\end{equation}
\begin{equation}
    L 
    \leq 
    \lim_{x \to a} g(x) 
    \leq 
    L 
\end{equation}
O que implica que
\begin{equation*}
    \lim_{x \to a} g(x) = L
\end{equation*}


\section{Lei de L'Hôspital}

sejam \(f\) e \(g\) funções diferenciáveis em um intervalo aberto \(I\) e algum \(c \in I\). Se:
\begin{enumerate}
    \item{ 
        \[
            \lim_{x \to c} f(x) = \lim_{x \to c} g(x)  = 0 
        \]
    }
    Ou
    \item{ 
        \[
            \lim_{x \to c} f(x) = \lim_{x \to c} g(x)  = \pm \infty 
        \]
    }
\end{enumerate}
ou seja, se tivermos que:
\begin{equation*}
    \lim_{x \to c} \frac{f(x)}{g(x)} = \frac{f(c)}{g(c)} = \frac{0}{0} 
    \quad \text{ou} \quad
    \lim_{x \to c} \frac{f(x)}{g(x)} = \frac{f(c)}{g(c)} = \pm \frac{\infty}{\infty}
\end{equation*}
Então
\begin{equation}
    \lim_{x \to c} \frac{f(x)}{g(x)} = 
    \lim_{x \to c} \frac{f'(x)}{g'(x)} 
\end{equation}