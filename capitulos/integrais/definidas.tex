\chapter{Integrais definidas}

\section{Teorema fundamental do Calculo}

\subsection{Teorema I}

\begin{theorem}
    Seja \(f\) contínua em \([a,b]\) e diferenciável em \((a,b)\). então a função \(g\)definida por
    
    \begin{equation}
        g(x) = \int_{a}^{x} f(t) dt ,\quad a \leq x \leq b
    \end{equation}
    Então
    \begin{equation}
        g'(x) = f(x)
    \end{equation}    
\end{theorem}



\subsection{Teorema II}
\begin{theorem}
    \begin{equation}
        \int_{a}^{b} f(x) dx = F(a) - F(b),\quad F'(x) = f(x)
    \end{equation}
\end{theorem}


\section{Propriedades}

\subsection{Igualdades}

Assuma \( a > b\). Temos:

\begin{flalign}
    \int_b^a f(x)\; dx = - \int_a^b f(x) &&
\end{flalign}

\begin{flalign}
    \int_a^a f(x)\; dx = 0 &&
\end{flalign}

\begin{flalign}
    \int_b^a c\; dx = c(b - a) &&
\end{flalign}

\begin{flalign}
    \int_b^a [f(x) \pm g(x) ] dx = 
    \int_b^a f(x)\; dx \pm 
    \int_b^a g(x)\; dx  &&
\end{flalign}

\begin{flalign}
    \int_b^a c.f(x) dx = c\int_b^a f(x)\; dx &&
\end{flalign}

\begin{flalign}
    \int_a^c f(x)\; dx + \int_c^b f(x)\; dx = \int_a^b f(x)\;dx &&
\end{flalign}

\subsection{Desigualdades}

\begin{flalign}
    \text{Se \(f(x) > 0 \) para \( a \leq x \leq b \implies\)} &&
\end{flalign}
\begin{equation}
    \int_a^b f(x)\; dx > 0
\end{equation}

\begin{flalign}
    \text{Se \(f(x) > g(x) \) para \( a \leq x \leq b \implies\)} &&
\end{flalign}
\begin{equation}
    \int_a^b f(x)\; dx > \int_a^b g(x)\; dx 
\end{equation}

\begin{flalign}
    \text{Se \( m \leq f(x) \leq M \) para \( a \leq x \leq b \implies\)} &&
\end{flalign}
\begin{equation}
    m(b - a ) \leq \int_a^b f(x)\; dx \leq M(b - a)
\end{equation}