\chapter{Continuidade}

\section{Continuidades}

\subsection{Continuidade num ponto}
\begin{definition}
    Dizemos que \(f\) é continua em um ponto \(a\) se:
    
    \begin{equation}
       \lim_{x \to a} f(x) = f(a) 
    \end{equation}
    
\end{definition}

\subsection{Continuidade a direita}

\begin{definition}
    Se
    \begin{equation}
       \lim_{x \to a^+} f(x) = f(a)  
    \end{equation}
    Então f é continua a direita do ponto a
\end{definition}

\subsection{Continuidade a esquerda}

\begin{definition}
    Se
    \begin{equation}
       \lim_{x \to a^-} f(x) = f(a)  
    \end{equation}
    Então f é continua a esquerda do ponto a
\end{definition}

\subsection{Continuidade em intervalos}

\begin{definition}
    Dizemos que \(f\) é continua em um intervalo aberto \( I = (a, b)\) se:
    
    \begin{equation}
       \lim_{x \to c} f(x) = f(c),\quad \forall c \in I
    \end{equation}
    
\end{definition}

\begin{definition}
    Dizemos que \(f\) é continua em um intervalo fechado \( I = [a, b]\) se:
    
    \begin{equation}
       \lim_{x \to c} f(x) = f(c),\quad \forall c \in I
    \end{equation}
    
\end{definition}


\section{Descontinuidades}

\subsection{Descontinuidades removíveis}

\begin{definition}
    Se
    \begin{equation*}
        \exists L; \quad \lim_{x \to a} f(x) = L
    \end{equation*}
    E
    \begin{center}
        \(f(a) \neq L\) ou \(f(a)\) não existe
    \end{center}
    Então a descontinuidade é removível se redefinirmos \(f(a)\) como sendo \(L\)
\end{definition}

\subsection{Descontinuidades essenciais}


\begin{definition}
    Se
    \begin{equation*}
        \exists L; \quad \lim_{x \to a} f(x) = L
    \end{equation*}
    E
    \begin{center}
        \(f(a) \neq L\) ou \(f(a)\) não existe
    \end{center}
    Então a descontinuidade é removível se redefinirmos \(f(a)\) como sendo \(L\)
\end{definition}

\subsubsection{Descontinuidade de pulo}

\begin{definition}
    Se
    \begin{equation}
       \lim_{x \to a^+} f(x) \neq \lim_{x \to a^-}
    \end{equation}
    Então dizemos que a função tem descontinuidade do tipo pulo
\end{definition}

\subsubsection{Descontinuidade infinita}

\begin{definition}
    Se
    \begin{equation}
       \lim_{x \to a^{\pm}} f(x) = \pm \infty
    \end{equation}
    Então dizemos que a função tem descontinuidade do tipo infinito
\end{definition}


\section{Regras de continuidade}

Sejam \(f\) e \(g\) funções contínuas em um intervalo \(I\) e \(c\) uma constante qualquer.
Vale que são continuas as funções:

\begin{flalign}
    (f \pm g)(x) &&
\end{flalign}
\begin{flalign}
    (c.f)(x) &&
\end{flalign}
\begin{flalign}
    (f.g)(x) &&
\end{flalign}
\begin{flalign}
    \left(
        \frac{f}{g}
    \right)
    (x), \; g(x) \neq 0, \forall x \in I &&
\end{flalign}
\begin{flalign}
    (f \circ g)(x), \text{se f contínua em g(x)}, \forall x \in I &&
\end{flalign}
\begin{flalign}
    (f^{-1})(x),\; \text{se f é injetora}  &&
\end{flalign}
Também são continuas as funções
\begin{enumerate}
    \item Polinomiais
    \item Racionais
    \item Exponenciais e Logarítmicas
    \item Trigonométricas
\end{enumerate}