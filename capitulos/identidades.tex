\chapter{Identidades trigonométricas}

\begin{flalign}
    sen^2(x) + cos^2(x) = 1 &&
\end{flalign}
\begin{flalign}
    sec^2(x) = 1 + tan^2(x) &&
\end{flalign}
\begin{flalign}
    tan^2(x) = sec^2(x) - 1 &&
\end{flalign}
\begin{flalign}
    sen^2(x) = \frac{1}{2} ( 1 - cos(2x)) &&
\end{flalign}
\begin{flalign}
    cos^2(x) = \frac{1}{2} ( 1 - sen(2x)) &&
\end{flalign}

\begin{flalign}
    cos(x)sen(x) = \frac{1}{2} sen(2x) &&
\end{flalign}

\begin{flalign}
    sen(a)cos(b) = 
    \frac{1}{2} 
    [
    sen(a - b)
    +
    sen(a + b
    ] &&
\end{flalign}

\begin{flalign}
    sen(a)sen(b) = 
    \frac{1}{2} 
    [
    cos(a - b)
    -
    cos(a + b
    ] &&
\end{flalign}


\begin{flalign}
    sen(a)sen(b) = 
    \frac{1}{2} 
    [
    cos(a - b)
    +
    cos(a + b
    ] &&
\end{flalign}