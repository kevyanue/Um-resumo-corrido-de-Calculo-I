\documentclass[14pt]{extreport}
\usepackage{amsfonts}
\usepackage{amsthm}
\usepackage{amsmath}
\usepackage{amssymb}
\usepackage{array}
\usepackage{relsize}
\usepackage{graphicx}
\usepackage[thinlines]{easytable}
\usepackage[utf8]{inputenc}
\usepackage[portuguese]{babel}
\usepackage{hyperref}

\title{Resumo corrido de Calculo - aceite se quiser}
\author{Kevyan Uehara}
\date{Novembro 2019}

\newtheorem{theorem}{Teorema}
\newtheorem{corollary}{Corolário}[theorem]
\newtheorem{lemma}{Lema}
\theoremstyle{definition}
\newtheorem{definition}{Definição}
\newtheorem{proposition}{Proposição}
\newtheorem*{remark}{Observação}
\newtheorem{method}{Método}
\newtheorem{example}{Exemplo}


\begin{document}

\maketitle

\chapter*{Prefácio}

\begin{enumerate}
    \item Não tem demonstrações
    \item Não tem gráficos
    \item Não tem todo o conteúdo
    \item Não tem exemplos
    \item Não tem aplicações
    \item Contém: regras, formulas
    \item mais tarde: macetes e mnemônicos
    \item Boa sorte
\end{enumerate}

\tableofcontents

\chapter{Identidades trigonométricas}

\begin{flalign}
    sen^2(x) + cos^2(x) = 1 &&
\end{flalign}
\begin{flalign}
    sec^2(x) = 1 + tan^2(x) &&
\end{flalign}
\begin{flalign}
    tan^2(x) = sec^2(x) - 1 &&
\end{flalign}
\begin{flalign}
    sen^2(x) = \frac{1}{2} ( 1 - cos(2x)) &&
\end{flalign}
\begin{flalign}
    cos^2(x) = \frac{1}{2} ( 1 - sen(2x)) &&
\end{flalign}

\begin{flalign}
    cos(x)sen(x) = \frac{1}{2} sen(2x) &&
\end{flalign}

\begin{flalign}
    sen(a)cos(b) = 
    \frac{1}{2} 
    [
    sen(a - b)
    +
    sen(a + b
    ] &&
\end{flalign}

\begin{flalign}
    sen(a)sen(b) = 
    \frac{1}{2} 
    [
    cos(a - b)
    -
    cos(a + b
    ] &&
\end{flalign}


\begin{flalign}
    sen(a)sen(b) = 
    \frac{1}{2} 
    [
    cos(a - b)
    +
    cos(a + b
    ] &&
\end{flalign}

\chapter{Limites}

\section{Regras}

Seja \(C\) uma constante qualquer e suponha que 
\noindent
\begin{equation*}
    \exists L_f ; \lim_{x \to a} f(x) = L_f
    \quad
    e
    \quad
    \exists L_g ; \lim_{x \to a} f(x) = L_g
\end{equation*}
    
\begin{flalign}
    \lim_{ x \to a } 
        \left[
            f(x) \pm g(x)
        \right]
    = 
        \left[
            \lim_{x \to a} f(x)
        \right]
            \pm 
        \left[
            \lim_{x \to a} f(x)]
        \right]
    = 
        L_f + L_g &&
\end{flalign}

\begin{flalign}
    \lim_{ x \to a } 
        \left[
            C.f(x)
        \right]
    = 
        C . \left[
                \lim_{x \to a} f(x)
            \right]
    = 
        C.L_f &&
\end{flalign}

\begin{flalign}
    \lim_{ x \to a } 
        \left[
            f(x) . g(x)
        \right]
    = 
        \left[
            \lim_{x \to a} f(x)
        \right]
        .
        \left[
            \lim_{x \to a} g(x)
        \right]
    = 
        L_f . L_g &&
\end{flalign}

\begin{flalign}
    \lim_{ x \to a } 
        \left[
            \frac{f(x)}{g(x)} 
        \right]
    = 
        \left[
            \frac{
                \lim_{x \to a} f(x)
            }{
                \lim_{x \to a} g(x)
            }    
        \right]
    = 
        \left[
            \frac{
                L_f 
            }{
                L_g
            }
        \right]        
    ,
        \lim_{x \to a} g(x) \neq 0 &&
\end{flalign}

\begin{flalign}
    \lim_{ x \to a } 
        \left[
            f(x)
        \right]^n
    = 
        \left[
            \lim_{x \to a} f(x)
        \right]^n
    = 
        [L_f]^n &&
\end{flalign}

\subsection{Limite da composta}

Sejam \(f\) e \(g\) duas funções tais que
\begin{equation*}
    \lim_{x \to a} g(x) = L_g 
    \quad e \quad
    \lim_{x \to a} f(x) = L_f
\end{equation*}
Se, para \(x\) próximos de a e \(x \neq a\) temos
\begin{equation}
    g(x) \in Df \quad e \quad g(x) \neq L_g
    \implies \lim_{x \to a} (f \circ g)(x) = L_f 
\end{equation}

\subsection{Limite da composta (feat. continuidade)}
Se
\begin{enumerate}
    \item \(\lim_{x \to a} g(x) = L_g\)
    \item \(f\) \text{contínua em} \(L_g\)
    \item \(g(x) \in Df\) para \(x\) próximos de \(a\) e \(x \neq a\)
\end{enumerate}
Então
\begin{equation}
    \lim_{x \to a} f(g(x)) 
    = 
    f \left(
        \;
        \lim_{x \to a} g(x)  
        \;
      \right)
    = f (L_g)  
\end{equation}
a função \(f\) ``saí pra fora'' do limite
\subsection{Mudança de variável}
Considerando o limite:
\begin{equation*}
    \lim_{x \to a} f(g(x))
\end{equation*}
Se definirmos \(u = g(x)\),
temos que
\begin{equation}\label{md}
    x \to a,\, u = g(x) \to b
\end{equation}
O que nos da que
\begin{equation*}
    \lim_{x \to a} f(g(x)) 
    \stackrel{\ref{md}}{=}
    \lim_{u \to b} f(u) 
\end{equation*}
Ou seja, nosso limite dado em função de \(x\) está agora em função de \(u\)
\section{Limites fundamentais}

\begin{flalign}
    \lim_{x \to 0} \frac{\sin{x}}{x} = 1 &&
\end{flalign}

\begin{flalign}
    \lim_{x \to \pm \infty}\left(1+\frac{1}{x}\right)^{x} = e &&
\end{flalign}

\begin{flalign}
    \lim_{x \to 0}\frac{a^{x}-1}{x} = \ln{a} &&
\end{flalign}

\begin{flalign}
    \lim_{x \to 0}\frac{a^{x}-1}{x}=\ln a &&
\end{flalign}

\begin{flalign}
    \lim_{x \to 0^+}\frac{1}{x} = +\infty &&
\end{flalign}

\begin{flalign}
    \lim_{x \to 0^-}\frac{1}{x} = -\infty &&
\end{flalign}

\begin{flalign}
    \lim_{x \to \pm \infty}\frac{1}{x} = 0 &&
\end{flalign}


\subsection{Outros limites}

\begin{flalign}
    \lim_{x\to \pm \infty}\left(1+\frac{a}{x}\right)^{bx}= e^{ab} &&
\end{flalign}

\begin{flalign}
    \lim_{x\to \pm \infty} a^{\frac{1}{x}} = 1 &&
\end{flalign}

\begin{flalign}
    \lim_{x\to \pm \infty} \frac{\ln{n}}{n} = 0 &&
\end{flalign}

\begin{flalign}
    \lim_{x \to 0} \frac{1 - \cos{x}}{x} = 0 &&
\end{flalign}

\section{Regras do infinito}
Aqui considere:
\begin{enumerate}
    \item L é um numero qualquer \textbf{Diferente de 0}
    \item \(L^+\) é um numero qualquer \textbf{estritamente} positivo
    \item \(L^-\) é um numero qualquer \textbf{estritamente} negativo
    \item \(\pm \infty\) como algum limite de alguma função \(f\)
\end{enumerate}
Temos que:
\begin{flalign}
    \infty + \infty = \infty &&
\end{flalign}
\begin{flalign}
    \infty . \infty = \infty &&
\end{flalign}
\begin{flalign}
    L + \infty = \infty &&
\end{flalign}
\begin{flalign}
    \infty - L = \infty &&
\end{flalign}
\begin{flalign}
    L^+ . \infty = \infty &&
\end{flalign}
\begin{flalign}
    L^- . \infty = -\infty &&
\end{flalign}
\begin{flalign}
    L^+ . -\infty = -\infty &&
\end{flalign}
\begin{flalign}
    L^- . -\infty = +\infty &&
\end{flalign}
\begin{flalign}
    -\infty - \infty = -\infty &&
\end{flalign}
\begin{flalign}
    -\infty . (- \infty) = +\infty &&
\end{flalign}

\subsection{Indeterminações com infinitos}

Assuma o mesmo que a seção anterior.
As seguintes expressões são indet.:

\begin{flalign}
    \infty - \infty &&
\end{flalign}
\begin{flalign}
    0 . \infty &&
\end{flalign}
\begin{flalign}
    \frac{\pm\infty}{\pm\infty} &&
\end{flalign}

\begin{flalign}
    \frac{0}{0} &&
\end{flalign}

\begin{flalign}
    1^{\infty} &&
\end{flalign}
\begin{flalign}
    \infty^{0} &&
\end{flalign}
\begin{flalign}
    0^{0} &&
\end{flalign}

\section{Teorema que antecede o confronto}
se temos que 
\begin{equation*}
    f(x) \leq g(x)
\end{equation*}
então
\begin{equation}
    \lim_{x \to a} f(x) 
    \leq 
    \lim_{x \to a} g(x)
\end{equation}
(se os limites existirem)
\section{Teorema do confronto}
sejam \(f\), \(g\) e \(h\) funções. Se temos que:
\begin{enumerate}
    \item \[f(x) \leq g(x) \leq h(x)\]
    \item \[\lim_{x \to a} f(x) = L = \lim_{x \to a} h(x)\] 
\end{enumerate}
Temos que
\begin{equation}
    \lim_{x \to a} f(x)
    \leq 
    \lim_{x \to a} g(x) 
    \leq 
    \lim_{x \to a} h(x) 
\end{equation}
\begin{equation}
    L 
    \leq 
    \lim_{x \to a} g(x) 
    \leq 
    L 
\end{equation}
O que implica que
\begin{equation*}
    \lim_{x \to a} g(x) = L
\end{equation*}


\section{Lei de L'Hôspital}

sejam \(f\) e \(g\) funções diferenciáveis em um intervalo aberto \(I\) e algum \(c \in I\). Se:
\begin{enumerate}
    \item{ 
        \[
            \lim_{x \to c} f(x) = \lim_{x \to c} g(x)  = 0 
        \]
    }
    Ou
    \item{ 
        \[
            \lim_{x \to c} f(x) = \lim_{x \to c} g(x)  = \pm \infty 
        \]
    }
\end{enumerate}
ou seja, se tivermos que:
\begin{equation*}
    \lim_{x \to c} \frac{f(x)}{g(x)} = \frac{f(c)}{g(c)} = \frac{0}{0} 
    \quad \text{ou} \quad
    \lim_{x \to c} \frac{f(x)}{g(x)} = \frac{f(c)}{g(c)} = \pm \frac{\infty}{\infty}
\end{equation*}
Então
\begin{equation}
    \lim_{x \to c} \frac{f(x)}{g(x)} = 
    \lim_{x \to c} \frac{f'(x)}{g'(x)} 
\end{equation}
    
\chapter{Continuidade}

\section{Continuidades}

\subsection{Continuidade num ponto}
\begin{definition}
    Dizemos que \(f\) é continua em um ponto \(a\) se:
    
    \begin{equation}
       \lim_{x \to a} f(x) = f(a) 
    \end{equation}
    
\end{definition}

\subsection{Continuidade a direita}

\begin{definition}
    Se
    \begin{equation}
       \lim_{x \to a^+} f(x) = f(a)  
    \end{equation}
    Então f é continua a direita do ponto a
\end{definition}

\subsection{Continuidade a esquerda}

\begin{definition}
    Se
    \begin{equation}
       \lim_{x \to a^-} f(x) = f(a)  
    \end{equation}
    Então f é continua a esquerda do ponto a
\end{definition}

\subsection{Continuidade em intervalos}

\begin{definition}
    Dizemos que \(f\) é continua em um intervalo aberto \( I = (a, b)\) se:
    
    \begin{equation}
       \lim_{x \to c} f(x) = f(c),\quad \forall c \in I
    \end{equation}
    
\end{definition}

\begin{definition}
    Dizemos que \(f\) é continua em um intervalo fechado \( I = [a, b]\) se:
    
    \begin{equation}
       \lim_{x \to c} f(x) = f(c),\quad \forall c \in I
    \end{equation}
    
\end{definition}


\section{Descontinuidades}

\subsection{Descontinuidades removíveis}

\begin{definition}
    Se
    \begin{equation*}
        \exists L; \quad \lim_{x \to a} f(x) = L
    \end{equation*}
    E
    \begin{center}
        \(f(a) \neq L\) ou \(f(a)\) não existe
    \end{center}
    Então a descontinuidade é removível se redefinirmos \(f(a)\) como sendo \(L\)
\end{definition}

\subsection{Descontinuidades essenciais}


\begin{definition}
    Se
    \begin{equation*}
        \exists L; \quad \lim_{x \to a} f(x) = L
    \end{equation*}
    E
    \begin{center}
        \(f(a) \neq L\) ou \(f(a)\) não existe
    \end{center}
    Então a descontinuidade é removível se redefinirmos \(f(a)\) como sendo \(L\)
\end{definition}

\subsubsection{Descontinuidade de pulo}

\begin{definition}
    Se
    \begin{equation}
       \lim_{x \to a^+} f(x) \neq \lim_{x \to a^-}
    \end{equation}
    Então dizemos que a função tem descontinuidade do tipo pulo
\end{definition}

\subsubsection{Descontinuidade infinita}

\begin{definition}
    Se
    \begin{equation}
       \lim_{x \to a^{\pm}} f(x) = \pm \infty
    \end{equation}
    Então dizemos que a função tem descontinuidade do tipo infinito
\end{definition}


\section{Regras de continuidade}

Sejam \(f\) e \(g\) funções contínuas em um intervalo \(I\) e \(c\) uma constante qualquer.
Vale que são continuas as funções:

\begin{flalign}
    (f \pm g)(x) &&
\end{flalign}
\begin{flalign}
    (c.f)(x) &&
\end{flalign}
\begin{flalign}
    (f.g)(x) &&
\end{flalign}
\begin{flalign}
    \left(
        \frac{f}{g}
    \right)
    (x), \; g(x) \neq 0, \forall x \in I &&
\end{flalign}
\begin{flalign}
    (f \circ g)(x), \text{se f contínua em g(x)}, \forall x \in I &&
\end{flalign}
\begin{flalign}
    (f^{-1})(x),\; \text{se f é injetora}  &&
\end{flalign}
Também são continuas as funções
\begin{enumerate}
    \item Polinomiais
    \item Racionais
    \item Exponenciais e Logarítmicas
    \item Trigonométricas
\end{enumerate}
    
\chapter{Derivadas}

\begin{equation}
f'(x) 
=
\lim_{x \to 0} \frac{
    f(x + h) - f(x)
}{
    h
}
\end{equation}
Ou
\begin{equation}
f'x 
= 
\lim_{x \to a} \frac{
    f(x) - f(a)
}{
    x - a
}    
\end{equation}


\subsection*{Notações}

temos as seguintes notações para a primeira derivada e as derivadas superiores:
\begin{flalign}
    f'(x), f''(x), f'''(x), ... &&
\end{flalign}
\begin{flalign}
    \frac{d}{dx} f, \frac{d^2}{dx^2} f, \frac{d^3}{dx^3}f , ... &&
\end{flalign}
\begin{flalign}
    D^1 f(x), D^2 f(x), D^3 f(x), ... &&
\end{flalign}
\begin{flalign}
    f^{(1)}, f^{(2)}, f^{(3)}, ... &&
\end{flalign}
OBS: Note que a \(n\) derivada \(f^{\textbf{(n)}}(x)\) é diferente da \(n\) composta \(f^{\textbf{n}}(x)\) apesar das notações serem parecidas. 

\section{Diferenciabilidade e continuidade}

\begin{definition}
    \(f\) diferenciável em \(a\) se \(f'(a)\) existe    
\end{definition}

\begin{theorem}
    \begin{equation}
        \text{f diferenciável} \implies \text{f continua}
    \end{equation}
    
    Não vale a volta.
\end{theorem}

\begin{theorem}
    \begin{equation}
        \text{f descontinua em a} \implies \text{f não-diferenciável em a}  
    \end{equation}
    
    Não vale a volta.
\end{theorem}

\section{reta rangente}

A reta \(r\) tangente a um ponto \((a, f(a))\)
é da por

\begin{equation}
y = f'(a)(x - a) + f(a)    
\end{equation}


\section{Regras de derivação}

\textbf{multiplicação por escalar}

\begin{flalign}
    \frac{d}{dx} \left( cf(x) \right) = c \frac{d}{dx} f(x)    &&
\end{flalign}
\\
\textbf{Soma}
\\
\begin{flalign}
    \frac{d}{dx} (f + g)(x) =
    \frac{d}{dx} f(x) + \frac{d}{dx} g(x) &&
\end{flalign}
\\
\textbf{Multiplicação}
\\
\begin{flalign}
    \frac{d}{dx} (fg)(x) = f'(x)g(x) + f(x)g'(x) &&
\end{flalign}
\\
\textbf{Divisão}
\\
\begin{flalign}
    \frac{d}{dx} \left( \frac{f}{g} \right) (x)
    = \frac{f'(x)g(x) - f(x)g'(x)}{g(x)^2} &&
\end{flalign}
\\
\textbf{Composta}
\\
\begin{flalign}
    \frac{d}{dx} (f \circ g)(x) = (f' \circ g)(x) g'(x) &&
\end{flalign}
\\
\textbf{Inversa}
\\
\begin{flalign}
    \frac{d}{dx} f^{-1}(y) = \frac{1}{(f' \circ f^{-1})(y)} &&
\end{flalign}

\section{Tabela de derivadas}

\subsection{Derivadas simples}

\begin{flalign}
    \frac{d}{dx} c = 0 &&
\end{flalign}
\begin{flalign}
    \frac{d}{dx} x = 1 &&
\end{flalign}
\begin{flalign}
    \frac{d}{dx} x^{n} = nx^{n-1} &&
\end{flalign}
\begin{flalign}
    \frac{d}{dx} \frac{1}{x^n} =
    \frac{d}{dx} (x^{-c}) = -\frac{c}{x^{c+1}} &&
\end{flalign}
\begin{flalign}
    \frac{d}{dx} \sqrt{x} = x^{\frac{1}{2}} = -\frac{1}{2} x^{-\frac{1}{2}} = -\frac{1}{2\sqrt{x}} &&
\end{flalign}
\begin{flalign}
    &\frac{d}{dx} e^x = e^x &&
\end{flalign}
\begin{flalign}
    &\frac{d}{dx} e^{cx} = ce^{cx} &&
\end{flalign}
\begin{flalign}
    &\frac{d}{dx} \log_a x = \frac{1}{ln(a)}. \frac{1}{x} &&
\end{flalign}
\begin{flalign}
    &\frac{d}{dx} \ln{x} = \frac{1}{x} &&
\end{flalign}
\begin{flalign}
    &\frac{d}{dx} a^x = a^x ln(a) &&
\end{flalign}

\subsection{Derivadas de trigonométricas}

\textbf{Macete}

se tem Derivada \(cos\) aparece sinal de \(-\) e derivadas do \(sen\) não tem sinal adicional.

e segue essa sequencia

\begin{equation}
      sen(x) \stackrel{D}{\longrightarrow} 
      cos(x) \stackrel{D}{\longrightarrow} 
    - sen(x) \stackrel{D}{\longrightarrow} 
    - cos(x) \stackrel{D}{\longrightarrow}
      sen(x) \stackrel{D}{\to} ...
\end{equation}

\begin{flalign}
    \frac{d}{dx} sen(x) = cos(x) &&
\end{flalign}

\begin{flalign}
    \frac{d}{dx} cos(x) = -sen(x) &&
\end{flalign}

\begin{flalign}
    \frac{d}{dx} -sen(x) = -cos(x) &&
\end{flalign}

\begin{flalign}
    \frac{d}{dx} -cos(x) = sen(x) &&
\end{flalign}

\begin{flalign}
    \frac{d}{dx} tan(x) = sec^2(x) &&
\end{flalign}

\begin{flalign}
    \frac{d}{dx} sec(x) = tan(x)sec(x) &&
\end{flalign}

\begin{flalign}
    \frac{d}{dx} cotan(x) = -cossec^2(x) &&
\end{flalign}

\begin{flalign}
    \frac{d}{dx} cosec(x) = -cotan(x)cosec(x) &&
\end{flalign}




\subsection{Derivadas de trigonométricas inversas}

Passo a passo geral, onde \(trig\) é uma função trigonométrica qualquer e \(arctrig\) é sua inversa
\begin{enumerate}
    \item estabeleça o domínio de \(trig\)
    \item note que \(y = trig(x) \iff arctrig(y) = x\)
    \item Diferencie implicitamente
    \item isole o \(y'\) como usualmente se faz e terá a derivada da inversa trigonométrica
\end{enumerate}

\begin{flalign}
    \frac{d}{dx} arcsen(x) = \frac{1}{\sqrt{1 - x^2}} &&
\end{flalign}

\begin{flalign}
    \frac{d}{dx} arccos(x) = - \frac{1}{\sqrt{1 - x^2}} &&
\end{flalign}

\begin{flalign}
    \frac{d}{dx} arctan(x) = \frac{1}{1 + x^2} &&
\end{flalign}

\begin{flalign}
    \frac{d}{dx} arccotan(x) = - \frac{1}{1 + x^2} &&
\end{flalign}

\begin{flalign}
    \frac{d}{dx} arcsec(x) = \frac{1}{x \sqrt{x^2 - 1}} &&
\end{flalign}

\begin{flalign}
    \frac{d}{dx} arccosec(x) = - \frac{1}{x \sqrt{x^2 - 1}} &&
\end{flalign}

\subsection{Derivadas de potencias}

\begin{flalign}
    \frac{d}{dx} x^x = x^x(ln(x) + 1),\quad para\; x>0 &&
\end{flalign}

\begin{flalign}
    \frac{d}{dx} g(x)^a = a.g(x)^{a-1}.g'(x) &&
\end{flalign}

\begin{flalign} 
    \frac{d}{dx} a^{g(x)} = ln(a).a^{g(x)}.g'(x) &&
\end{flalign}

\begin{flalign}
    \frac{d}{dx} f(x)^{g(x)} = \frac{d}{dx}  \, e^{g(x).ln(f(x))} &&
\end{flalign}

\begin{flalign}
     = f(x)^{g(x)} . g'(x) . ln (f(x)) + f(x)^{g(x) - 1}.g(x).f'(x) &&
\end{flalign}

\chapter{Trigonométricas hiperbólicas}
\section{Trigonométricas hiperbólicas}

\begin{flalign}
     senh(x) = \frac{e^x - e^{-x}}{2} &&
\end{flalign}

\begin{flalign}
     cosh(x) = \frac{e^x + e^{-x}}{2} &&
\end{flalign}
As demais funções trigonométricas são definidas como usualmente se faz, mas com \(senh\) e \(cosh\) no lugar de \(sen\) e \(cos\)

\section{Identidades trigonométricas hiperbólicas}

\begin{flalign}
    senh(x) = -senh(x), \quad \text{(senh é impar)} &&
\end{flalign}
\begin{flalign}
    cosh(-x) = cosh(x), \quad \text{(cosh é par)} &&
\end{flalign}
\begin{flalign}
    cosh^2(x) -senh^2(x) = 1 &&
\end{flalign}
\begin{flalign}
    1 - tanh^2(x) = sech^2(x) &&
\end{flalign}
\begin{flalign}
    senh(a + b) = senh(a)cosh(b) + cosh(a)senh(b) &&
\end{flalign}
\begin{flalign}
    cosh(a + b) = cosh(a)cosh(b) + senh(a)senh(b) &&
\end{flalign}

\section{Derivadas de trigonométricas hiperbólicas}

\begin{flalign}
    \frac{d}{dx} senh(x) = cosh(x) &&
\end{flalign}
\begin{flalign}
    \frac{d}{dx} cosh(x) = senh(x) &&
\end{flalign}
\begin{flalign}
    \frac{d}{dx} tanh(x) = sech^2(x) &&
\end{flalign}
\begin{flalign}
    \frac{d}{dx} sech(x) = - tangh(x) sech^2(x) &&
\end{flalign}
\begin{flalign}
    \frac{d}{dx} cotangh(x) = - cossech^2(x) &&
\end{flalign}
\begin{flalign}
    \frac{d}{dx} cossech(x) = - cotanh(x) cossec(x) &&
\end{flalign}

\section{Derivadas de inversa trigonométricas hiperbólicas}

\begin{flalign}
    \frac{d}{dx} arcsenh(x) = \frac{1}{\sqrt{1 + x^2}} &&
\end{flalign}

\begin{flalign}
    \frac{d}{dx} arccosh(x) = \frac{1}{\sqrt{-1 + x^2}} &&
\end{flalign}

\begin{flalign}
    \frac{d}{dx} arccosh(x) = \frac{1}{\sqrt{-1 + x^2}} &&
\end{flalign}

\begin{flalign}
    \frac{d}{dx} arctanh(x) = \frac{1}{1 - x^2} &&
\end{flalign}


\chapter{Teoremas sobre funções}

\section{Teorema do Valor intermediário}

Seja \(f\) uma função contínua no intervalo \(I = [a,b]\), \(N \in I\) um numero qualuer e \(f(a) \neq f(b)\).
Então existe \(c \in I\) tal que f(c) = N
    
\section{Teorema do anulamento (Teorema de Bolzano)}

Seja \(f\) uma função contínua no intervalo \(I = [a,b]\) e \(a \leq 0 \leq b\).
Então existe \(c \in I\) tal que f(c) = 0

\section{Teorema do valor Extremo (Teorema de \textit{Weistrass})}

\begin{theorem}
    Seja f continua em um intervalo fechado \([a,b]\).
    Então existe um \(M\) máximo e \(m\) mínimo contido em \([a,b]\)
\end{theorem}

\section{Teorema de Fermat}

\begin{theorem}
    Se f tiver um máximo ou mínimo local em \(c\), então \(c\) é um ponto crítico.
\end{theorem}

\section{Teorema de Rolle}

\begin{theorem}
    Seja \(f\) uma função tal que:
    \begin{enumerate}
        \item \(f\) contínua em \([a,b]\)
        \item \(f\) diferenciável em \((a,b)\)
        \item \(f(a)\) = \(f(b)\)
    \end{enumerate}
    Então
    \begin{equation}
        \exists c \in (a,b);\quad f'(c) = 0
    \end{equation}
    O que se traduz como: ``se os extremos são iguais então a derivada se anula em algum lugar''
\end{theorem}

\section{Teorema do Valor Médio}
\begin{theorem}
    seja \(f\) uma função. Se:
    \begin{enumerate}
        \item f contínua no intervalo \( [a,b] \)
        \item f diferenciável no intervalo \( (a,b) \)
    \end{enumerate}
    Então:
    
    \begin{equation}
        \exists c \in (a,b);\quad f'(c) = \frac{f(a) - f(b)}{a - b}
    \end{equation}    
\end{theorem}

\chapter{Aproximações}

\section{Aproximação linear}

Aproximamos uma função por sua reta tangente em algum ponto \(x\) por
\begin{equation}
    f(x) \approx L(x) = f'(a).(x-a) + f(a), \quad \text{para x próximos de a}    
\end{equation}
\begin{definition}
    E definimos \(L(x)\) como \textbf{linearização de \(f\) em \(a\)}
\end{definition}

\section{Diferenciais}

\begin{equation}
    \Delta y = f(x + \Delta x) - f(x)
\end{equation}

\begin{definition}[Diferencial dx]
    \begin{equation}
        dx = \Delta x
    \end{equation}
    
\end{definition}

\begin{definition}[Diferencial dy]
    \begin{equation}
        dy = L(x + \Delta x) - L(x)
    \end{equation}
    
\end{definition}

\begin{equation}
    \Delta y = f(x + \Delta x) - f(x) \approx L(x + \Delta x) - L(x)
\end{equation}

\begin{definition}[Erro relativo]
    O erro relativo \(E\) da aproximação é dado por
    \begin{equation}
        E = \frac{\Delta y}{y}    
    \end{equation}
\end{definition}

\section{Polinômio de Taylor}
O polinômio de taylor \(T\) de ordem \(K\) de uma função \(f\) no ponto \(a\) é dada por:
\begin{equation}
    T_K(f(a)) = \sum_{n=0}^{K} \frac{f^{(n)}(a)}{n!}(x - a)
\end{equation}
Onde \(f^{(n)}\) é a \textit{n-ésima} derivada de \(f\). Temos:
\begin{equation}
    T_K(f(a)) = \frac{f^{(0)}(a)}{0!}(x - a) +
           \frac{f^{(1)}(a)}{1!}(x - a) + ... +
           \frac{f^{(K)}(a)}{K!}(x - a)
\end{equation}
Para o valor especifico de \(N= \infty\) e \(f\) infinitamente diferenciável:
\begin{equation}
    T(f(a)) = \sum_{n=0}^{\infty} \frac{f^{(n)}(a)}{n!}(x - a)
\end{equation}
\begin{equation}
    T(f(a)) = \frac{f^{(0)}(a)}{0!}(x - a) +
           \frac{f^{(1)}(a)}{1!}(x - a) +
           \frac{f^{(2)}(a)}{2!}(x - a) + ...
\end{equation}


\chapter{Como conseguir informações sobre funções}

\section{Máximo e Mínimo}

\begin{definition}[ponto e valor de máximo absoluto]
    Seja \(c \in Df\). Dizemos que \(c\) é (\textit{ponto de}) máximo absoluto ou (\textit{ponto de}) máximo global se:
    
    \begin{equation}
        f(c) \geq f(x), \forall x \in Df
    \end{equation}
    e denominamos o número \(M = f(c)\) como o \textbf{valor} de máximo absoluto.
\end{definition}

\begin{definition}[ponto e valor de mínimo absoluto]
    A definição é análoga para mínimo absoluto e valor de mínimo absoluto
\end{definition}

\begin{definition}[ponto e valor de máximo local]
    Seja \(I \subset Df\) um intervalo aberto. Dizemos que \(c\) é (\textit{ponto de}) máximo local ou (\textit{ponto de}) máximo relativo se:
    
    \begin{equation}
        f(c) \geq f(x), \forall x \in I
    \end{equation}
    e denominamos o número \(M = f(c)\) como o \textbf{valor} de máximo local.
\end{definition}

\begin{definition}[ponto e valor de mínimo local]
    A definição é análoga para mínimo local e valor local
\end{definition}

\section{Ponto crítico}

\begin{definition}
    se \(f'(c) = 0\) ou \(f'(c)\) não existe então \(c\) é um ponto crítico.
\end{definition}

\section{Crescente e Decrescente}

\begin{theorem}
    Suponha \(f\) é diferenciável num intervalo aberto \(I\). Vale que:
    \begin{enumerate}
        \item Se \(f'(x) > 0, \forall x  \in I\) então f é crescente em I
        \item Se \(f'(x) < 0, \forall x  \in I\) então f é decrescente em I
        \item \(f'(x) = 0, \forall x  \in I\) então f é constante em I
    \end{enumerate}
\end{theorem}

\section{Assintotas}

\subsection{Horizontal}

\begin{definition}
    \begin{equation}
        \lim_{x \to \pm \infty} f(x) = L \implies \text{a reta} y = L \text{ é assintota horizontal de \(f\)}
    \end{equation}
\end{definition}

\subsection{Vertical}

\begin{definition}
    \begin{equation}
        \lim_{x \to a^{\pm}} f(x) = \pm \infty \implies \text{a reta} x = a \text{ é assintota vertical de \(f\)}
    \end{equation}
\end{definition}
Podemos achar as assintotas verticais da seguinte maneira
\begin{method}
    Seja \(P\) o conjunto dos pontos onde dado um numero qualquer \( a \in P\), temos que:
    \begin{enumerate}
        \item \(a \notin Df\) ou
        \item \(f\) descontinua em \(a\)
    \end{enumerate}
    Testando os limites laterais em cada um dos \( a \in P\) nos da todas as assintotas verticais.
\end{method}

\subsection{Oblíqua}

\begin{definition}
    \begin{equation}
        \lim_{x \to \pm \infty} [f(x) - (mx + b)] = 0, m \neq 0 \implies 
    \end{equation}
    \centering
    a reta \( y = mx + b\) é assintota oblíqua
\end{definition}
Para encontrar assintota obliqua:
\begin{method}
    Se existir \(m\) e \(b\) tal que:
    \begin{enumerate}
        \item 
            \begin{equation}
                \lim_{x \to \pm \infty}  \frac{f(x)}{x} = m
            \end{equation}
        \item
            \begin{equation}
                \lim_{x \to \pm \infty}  f(x) - mx = b
            \end{equation}
    \end{enumerate}
    Então
    \(y = mx + b\) é uma assintota obliqua
\end{method}

\section{Encontrar máx. ou mín.}

\subsection{intervalo fechado pra valores absoluto}

\begin{method}
    Seja \(f\) uma função contínua em um intervalo \([a,b]\).
    Podemos encontrar o máximo e mínimo absoluto da seguinte forma
    
    \begin{enumerate}
        \item encontre os valores de \(f\) nos pontos criticos em \((a,b)\)
        \item encontre os valores de \(f(a)\) e \(f(b)\) (extremos do intervalo)
        \item O menor valor dos números é o valor absoluto e o menor valor é o mínimo absoluto
        
    \end{enumerate}    
\end{method}



\subsection{Teste da primeira derivada}
\begin{method}
    Seja \(f\) uma função continua e \(c\) um ponto critico. Temos que:
    \begin{enumerate}
        \item se o sinal de \(f'\) mudar de positivo pra negativo em \(c\) \(\implies\) \(c\) é ponto de máximo local
        \item se o sinal de \(f'\) mudar de negativo pra positivo em \(c\) \(\implies\) \(c\) é ponto de máximo local
        \item se o sinal de \(f'\) não mudar, então \(c\) não é ponto de máximo nem de mínimo
    \end{enumerate}    
\end{method}



\subsection{Teste da segunda derivada}

\begin{method}
    Suponha \(f'(c) = 0\), ou seja, as raizes de \(f'(x)\) , e que \(f''(c)\) existe. Então temo que:
    
    \begin{enumerate}
        \item se \(f''(c) > 0 \implies\) f(c) é um valor de mínimo local
        \item se \(f''(c) < 0 \implies\) f(c) é um valor de máximo local
    \end{enumerate}    
\end{method}
O teste falha quando:
\begin{enumerate}
    \item \(\nexists f'(c)\)
    \item \(f''(c) = 0\)
    \item \(\nexists f''(c)\)
\end{enumerate}

\section{Concavidade}

\subsection{Teste de concavidade}

\begin{enumerate}
    \item \(f''(x) > 0,\, \forall x \in I \implies f\) em \(I\) é côncavo para cima
    \item \(f''(x) < 0,\, \forall x \in I \implies f\) em \(I\) é côncavo para baixo
\end{enumerate}

\begin{remark}
    Note que:\\
    \(f''(x) = 0,\, \forall x \in I \\
    \implies f'(x) = c \\
    \implies f(x) = cx + d \\ 
    \implies f\) é reta afim e não tem concavidade
\end{remark}

\subsection{Ponto de inflexão}

\begin{definition}
    Um ponto \(P\) onde a curva \(y = f(x)\) é chamado de \textit{Ponto de Inflexão} se:
    \begin{enumerate}
        \item se \(f\) for contínua em \(P\)
        \item e a curva mudar de concavidade em \(P\)
    \end{enumerate}
\end{definition}


\section{Construção do gráfico}


\begin{method}
    O algoritmo para construção do gráfico:
    \begin{enumerate}
        \item Determine o Domínio \(Df\) de \(f\)
        \item Determine onde existe intersecção com os eixos, os pontos:
            \begin{enumerate}
                \item onde \(f(x) = 0\)
                \item e o ponto \(f(0)\)
            \end{enumerate}
        \item Se existe Simetria.  Se ela é
            \begin{enumerate}
                \item Par: \(f(-x) = f(x), \forall x \in Df \implies\) gráfico simétrico em relação ao eixo \(y\)
                \item Impar: \(f(-x) = -f(x), \forall x \in Df \implies\) gráfico simétrico em relação a reta \(y = -x\)
                \item Periódico: \(f(x + p) = f(x), \forall x \in Df \implies\) o gráfico se repete em cada intervalo e comprimento \(p\) 
            \end{enumerate}
        \item Determinar as assintotas
            \begin{enumerate}
                \item Horizontais
                \item verticais
                \item Oblíquas
            \end{enumerate}
            \begin{remark}
                se existe assintota horizontal \(\implies\) não existe assintota obliquá
            \end{remark}
        \item Intervalos crescentes e decrescentes
        \item Pontos e valores de máximo e mínimo
        \item Concavidade e pontos de inflexão
        \item Esboçe a curva:
            \begin{enumerate}
                \item Marque as assintotas
                \item Marque todos os pontos que achou
                \item Marque os intervalos de concavidade e crescimento/decrescimento
                \item Desenha a curva atendendo aos que esta marcado
            \end{enumerate}
            
    \end{enumerate}
    
    
\end{method}

\chapter{Somatórios}

\section{formulas}

\begin{flalign}
    \sum_{i =1}^{n} c = c + c + c + ... + c  = nc &&
\end{flalign}

\begin{flalign}
    \sum_{i =1}^{n} i = 1 + 2 + 3 + ... + n  = \frac{n(n-1)}{2} &&
\end{flalign}

\begin{flalign}
    \sum_{i =1}^{n} i^2 = 1^2 + 2^2 + 3^2 + ... + n^2  = \frac{n(n+1)(2n-1)}{6} &&
\end{flalign}

\begin{flalign}
    \sum_{i =1}^{n} i^3 = 1^3 + 2^3 + 3^3 + ... + n^3  = \left[
        \frac{n(n-1)}{2}
    \right]^2 &&
\end{flalign}

\begin{flalign}
    \sum_{i =1}^{n} ai \pm bi = \sum_{i =1}^{n} ai \pm \sum_{i =1}^{n} bi &&
\end{flalign}

\begin{flalign}
    e^x = \sum_{i = 0}^{\infty} \frac{x^n}{n!} = \frac{x^0}{0!} + \frac{x^1}{1!} + \frac{x^2}{2!} + \frac{x^3}{3!} + \dots &&
\end{flalign}


\chapter{Integrais Indefinidas}

\section{Propriedades}

\subsection{Igualdades}

\begin{flalign}
    \int c f(x) dx = c\int f(x) dx &&
\end{flalign}

\begin{flalign}
    \int [g(x) \pm f(x)] dx = \int g(x) dx \pm \int f(x) dx  &&
\end{flalign}


\section{Tabela de integrais}

\subsection{Integrais simples}

\begin{flalign}
    \int k\; dx = kx + d &&
\end{flalign}

\begin{flalign}
    \int x^n\; dx = \frac{x^{n +1 }}{n+1} + d &&
\end{flalign}

\begin{flalign}
    \int \frac{1}{x}\; dx = ln(|x|) + d &&
\end{flalign}

\begin{flalign}
    \int a^x\; dx = \frac{a^x}{ln(a)} + d &&
\end{flalign}

\begin{flalign}
    \int e^x\; dx = e^x + d &&
\end{flalign}

\subsection{Integrais trigonométricas}

\begin{flalign}
    \int sen(x)\; dx = - cos(x) + d &&
\end{flalign}

\begin{flalign}
    \int cos(x)\; dx = sen(x) + d &&
\end{flalign}

\begin{flalign}
    \int sec^2(x)\; dx = tan(x) + d &&
\end{flalign}

\begin{flalign}
    \int sec(x)tan(x)\; dx = sec(x) + d &&
\end{flalign}

\begin{flalign}
    \int cossec^2(x)\; dx = -cotan(x) + d &&
\end{flalign}

\begin{flalign}
    \int \frac{1}{1 + x^2}\;dx = artctan(x) + d &&
\end{flalign}

\begin{flalign}
    \int \frac{1}{\sqrt{1 - x^2}}\; dx = arcsen(x) + d &&
\end{flalign}

\begin{flalign}
    \int cossec(x)cotan(x)\; dx = -cossec(x) + d &&
\end{flalign}

\begin{flalign}
    \int \frac{1}{x^2 + a^2} = \frac{1}{a} arctan \left( \frac{x}{a} \right) + d &&
\end{flalign}

\begin{flalign}
    \int tan(x)\; dx = ln(\,|sec(x)|\,) + d &&
\end{flalign}

\begin{flalign}
    \int sec(x)\; dx = ln(\,|sec(x)|\,) + tan(x) + d &&
\end{flalign}



\subsection{Integrais de trigonométricas hiperbólicas}

\begin{flalign}
    \int senh(x)\; dx = cosh(x) + d &&
\end{flalign}

\begin{flalign}
    \int cosh(x)\; dx = senh(x) + d &&
\end{flalign}


\chapter{Integrais definidas}

\section{Teorema fundamental do Calculo}

\subsection{Teorema I}

\begin{theorem}
    Seja \(f\) contínua em \([a,b]\) e diferenciável em \((a,b)\). então a função \(g\)definida por
    
    \begin{equation}
        g(x) = \int_{a}^{x} f(t) dt ,\quad a \leq x \leq b
    \end{equation}
    Então
    \begin{equation}
        g'(x) = f(x)
    \end{equation}    
\end{theorem}



\subsection{Teorema II}
\begin{theorem}
    \begin{equation}
        \int_{a}^{b} f(x) dx = F(a) - F(b),\quad F'(x) = f(x)
    \end{equation}
\end{theorem}


\section{Propriedades}

\subsection{Igualdades}

Assuma \( a > b\). Temos:

\begin{flalign}
    \int_b^a f(x)\; dx = - \int_a^b f(x) &&
\end{flalign}

\begin{flalign}
    \int_a^a f(x)\; dx = 0 &&
\end{flalign}

\begin{flalign}
    \int_b^a c\; dx = c(b - a) &&
\end{flalign}

\begin{flalign}
    \int_b^a [f(x) \pm g(x) ] dx = 
    \int_b^a f(x)\; dx \pm 
    \int_b^a g(x)\; dx  &&
\end{flalign}

\begin{flalign}
    \int_b^a c.f(x) dx = c\int_b^a f(x)\; dx &&
\end{flalign}

\begin{flalign}
    \int_a^c f(x)\; dx + \int_c^b f(x)\; dx = \int_a^b f(x)\;dx &&
\end{flalign}

\subsection{Desigualdades}

\begin{flalign}
    \text{Se \(f(x) > 0 \) para \( a \leq x \leq b \implies\)} &&
\end{flalign}
\begin{equation}
    \int_a^b f(x)\; dx > 0
\end{equation}

\begin{flalign}
    \text{Se \(f(x) > g(x) \) para \( a \leq x \leq b \implies\)} &&
\end{flalign}
\begin{equation}
    \int_a^b f(x)\; dx > \int_a^b g(x)\; dx 
\end{equation}

\begin{flalign}
    \text{Se \( m \leq f(x) \leq M \) para \( a \leq x \leq b \implies\)} &&
\end{flalign}
\begin{equation}
    m(b - a ) \leq \int_a^b f(x)\; dx \leq M(b - a)
\end{equation}
    
\chapter{Integrais Impróprias}

\section{Integrais impróprias no infinito}

\subsection{\([a,\infty)\)}

Integrais da forma

\begin{equation}
    \int_{a}^{\infty} f(t) \; dt
\end{equation}
definimos como
\begin{equation}
    \int_{a}^{\infty} f(t) \; dt = \lim_{x \to \infty} \int_{a}^{x} f(t) \; dt = L
\end{equation}

\subsection{\((-\infty, b]\)}
Integrais da forma

\begin{equation}
    \int_{-\infty}^{b} f(t) \; dt
\end{equation}
definimos como
\begin{equation}
    \int_{-\infty}^{b} f(t) \; dt = \lim_{x \to -\infty} \int_{x}^{b} f(t) \; dt = L
\end{equation}

\subsection{\((-\infty, \infty)\)}
Integrais da forma

\begin{equation}
    \int_{-\infty}^{\infty} f(t) \; dt
\end{equation}
definimos como
\begin{equation}
    \int_{-\infty}^{\infty} f(t) \; dt = 
        \lim_{x \to -\infty} \int_{x}^{b} f(t) \; dt +  
        \lim_{x \to \infty} \int_{b}^{x} f(t) \; dt = L
\end{equation}

\subsection{Convergentes e divergentes}

se o limite \(L\) existe nas integrais definidas, a integral é dita \textbf{Convergente}. Senão, a integral é dita \textbf{Divergente}.


\section{Propriedades de convergências e divergências}

\begin{enumerate}
    \item \( \int_{a}^{b} f(x)\) convergente \(> \int_{a}^{b} g(x) \implies \int_{a}^{b} g(x)\) convergente
    \item \(\int_{a}^{b} f(x)\) divergente \(< \int_{a}^{b} g(x) \implies \int_{a}^{b} g(x)\) divergente
\end{enumerate}

\section{Integrais com descontinuidade}

Se \(f\) integrável em \((a, b]\), definimos a integral como:
\begin{equation}
    \int_{a}^{b} f(x)dx = \lim_{\epsilon \to a^{+} } \int_{\epsilon}^{b} f(x)dx
\end{equation}
Se \(f\) integrável em \([a,b)\), definimos a integral como:
\begin{equation}
    \int_{a}^{b} f(x)dx = \lim_{\epsilon \to b^{-} } \int_{b}^{\epsilon} f(x)dx
\end{equation}
E se \(f\) integrável em \([a, b]\) com exceção de \(c \in [a,b]\) tal que \(a < c < b\) Podemos definir a integral da seguinte maneira:
\begin{equation}
    \int_{a}^{b} f(x)dx = 
        \lim_{\epsilon \to c^{-} } \int_{a}^{\epsilon} f(x)dx +
        \lim_{\epsilon \to c^{+} } \int_{\epsilon}^{b} f(x)dx 
\end{equation}


\chapter{Técnicas de integração}

\section{Integrais trigonométricas}

\subsection{\(sen^n(x).cos^m(x)\)}

\subsubsection{\(m\) é impar}

\begin{flalign}
    \int sen(x)^n.cos^m(x)\; dx &&
\end{flalign}
se \(m\) é impar \(\implies\) \(m = 2k + 1\)
\begin{flalign}
    \int sen(x)^n.cos^{2k + 1}(x)\; dx = &&
\end{flalign}
\begin{flalign}
    \int sen(x)^n.(cos^{2}(x))^kcos^{1}(x)\; dx = &&
\end{flalign}
Utilizando a identidade \( cos^2(x) = 1 - sen^2(x)\),
\begin{flalign}
    \int sen(x)^n.(1 - sen^2(x))^k cos^{1}(x)\; dx &&
\end{flalign}
Daí, tomando \( u = sen(x)\), temos \( du = cos(x)dx\) e segue que:
\begin{flalign}
    \int u^n.(1 - u^ 2)^k u\; du &&
\end{flalign}

\subsubsection{\(n\) é impar}

O processo é análogo, 

\begin{enumerate}
    \item Guarde um fator seno
    \item Use a identidade \( sen^2(x) = 1 - cos^2(x) \)
    \item Substitua \( u = cos(x) \)
    \item Resolva a integral
\end{enumerate}

\subsubsection{\(n\) e \(m\) são pares}

Utilizamos as identidades:
\begin{gather}
    sen^2(x) = \frac{1}{2} ( 1 - cos(2x)) \\
    cos^2(x) = \frac{1}{2} ( 1 - sen(2x)) \\
    cos(x)sen(x) = \frac{1}{2} sen(2x)
\end{gather}

\subsection{\(sen(mx).cos(nx)\)}

\begin{flalign}
    \int sen(mx).cos(nx)\; dx &&
\end{flalign}
Utilizamos a seguinte Identidade:

\begin{equation}
    sen(a)cos(b) = 
    \frac{1}{2} 
    [
    sen(a - b)
    +
    sen(a + b
    ]
\end{equation}

\begin{flalign}
    \int sen(mx).sen(nx)\; dx &&
\end{flalign}
Utilizamos a seguinte Identidade:

\begin{equation}
    sen(a)sen(b) = 
    \frac{1}{2} 
    [
    cos(a - b)
    -
    cos(a + b
    ]
\end{equation}

\begin{flalign}
    \int cos(mx).cos(nx)\; dx &&
\end{flalign}
Utilizamos a seguinte Identidade:

\begin{equation}
    sen(a)sen(b) = 
    \frac{1}{2} 
    [
    cos(a - b)
    +
    cos(a + b
    ]
\end{equation}

\subsection{\(tan^n(x).sec^m(x)\)}

\subsubsection{\(n\) é par}
\begin{enumerate}
    \item Guarde um fator \(sec^2(x)\)
    \item use que \(sec^2(x) = 1 + tan^2(x)\)
    \item substitua u = tan(x)
\end{enumerate}

\subsubsection{\(n\) é impar}

\begin{enumerate}
    \item Guarde um fator \(sec(x)tan(x)\)
    \item use que \(tan^2(x) = sec^2(x) - 1\)
    \item substitua u = sec(x)
\end{enumerate}

\subsubsection{\(n\) par e \(m\) ímpar}

\begin{enumerate}
    \item Expresse tudo em termos da secante
    \item Resolva as potencias da secante integrando por partes
\end{enumerate}

utilizando as integrais primitivas:
\begin{flalign}
    \int tan(x)\; dx = ln(|sec(x)|) + c
\end{flalign}
\begin{flalign}
    \int sec(x)\; dx = ln(|sec(x)|) + tan(x) + c
\end{flalign}

\section{Substituição trigonométrica}

\subsection{\(\sqrt{a^2 - x^2}\)}
Substitua:

\begin{equation}
    x = a.sen(\theta)
\end{equation}
com o triangulo representado por:
\begin{equation}
    a^2 = (\sqrt{a^2 - x^2})^2 + x^2
\end{equation}

\subsection{\(\sqrt{a^2 - x^2}\)}
Substitua:

\begin{equation}
    x = a.tan(\theta)
\end{equation}
com o triangulo representado por:
\begin{equation}
    (\sqrt{a^2 + x^2})^2 = a^2 + x^2
\end{equation}

\subsection{\(\sqrt{x^2 - a^2}\)}
Substituía:

\begin{equation}
    x = a.sec(\theta)
\end{equation}
com o triangulo representado por:
\begin{equation}
    x^2 = (\sqrt{x^2 - a^2})^2 + a^2 
\end{equation}


\section{Funções racionais}

Temos a integral 

\begin{equation}
    \int \frac{p(x)}{q(x)}\; dx
\end{equation}

Se \(grau(p) > grau(q)\) então efetuamos a divisão o que nos da temos:

\begin{equation}
    f(x) = \frac{p(x)}{q(x)} = \frac{r(x)}{q(x)} + s(x) 
\end{equation}
Onde \(grau(r) > grau(q)\). E nossa integral se transforma em:
\begin{equation}
    \int f(x) = \int \left( \frac{r(x)}{q(x)} + s(x)\; \right) dx
\end{equation}
E daí temos que:
\begin{equation}
    \int f(x) = \int \frac{r(x)}{q(x)}\; dx + \int s(x)\; dx 
\end{equation}
fatorando ambos os polinômios de \(q(x)\) temos que:

\begin{equation}
    \textstyle
    f(x) = \frac{
        r(x)
    }{
        (a_1x + b_1)^{n_1}
        \cdot ... \cdot 
        (a_kx + b_k)^{n_k}
        \cdot
        (d_1x^2 + e_1x + f_1)^{m_1}
        \cdot ... \cdot 
        (d_lx^2 + e_lx + f_l)^{m_l}
    }
\end{equation}
Onde cada \(a_ix + b_i, i = 1...k\), são equações lineares, e as equações \(d_jx^2 + e_jx + f_j, j = 1...l\), são equações de segundo grau irredutíveis \(( \Delta < 0 )\).
Queremos cada componente da fatoração como:
\\
\\
se a \(c_n(x)\) é a \textit{n-ésima} componente da fatoração de \(q(x)\) e tivermos \(c_n(x) = (a_ix + b_i)^k\) fatoramos como:
\begin{equation*}
    c_n(x) = 
        \frac{A_1}{(a_1x + b_1)^1} +
        \frac{A_2}{(a_2x + b_2)^2} + ... + 
        \frac{A_n}{(a_nx + b_n)^n}
\end{equation*}
que denotamos como:
\begin{equation}
    c_n(x) = \sum_{ i = 1}^{n} \frac{A_i}{(a_1x + b_1)^i}
\end{equation}
se a \(d_n(x)\) é a \textit{n-ésima} componente da fatoração de \(q(x)\) e tivermos \(d_n(x) = (a_jx^2 + b_jx + c_j)^l\) fatoramos como:
\begin{equation}
    d_n(x) = 
        \frac{B_1x + C_1}{(c_1x^2 + d_1x + e_1)^1} +
        \frac{B_2x + C_2}{(c_2x^2 + d_2x + e_2)^2} + ... + 
        \frac{B_nx + C_m}{(c_mx^2 + d_mx + e_m)^m}
\end{equation}
que denotamos como:
\begin{equation}
    d_n(x) = \sum_{j = 1}^{m} \frac{B_jx + C_j}{(c_jx^2 + d_jx + e_j)^j}
\end{equation}
Temos uma fatoração de \(f(x)\) de tal forma:
\begin{equation}
    \frac{p(x)}{q(x)} =
    \sum_{u = 1}^{v} 
        c_u(x) + 
    \sum_{w = 1}^{z} 
        d_w(x)
\end{equation}
teremos os coeficientes \(A_p, B_q, C_r\) resolvendo o sistema linear que decorre de:
\begin{equation}
    q(x) = 
        p(x)
            \left(
                \sum_{u = 1}^{v} 
            c_u(x) + 
                \sum_{w = 1}^{z} 
            d_w(x)
            \right)
\end{equation}
Descobertos os coeficientes das equações, teremos que:
\begin{equation}
    f(x) = 
        s(x) + \frac{p(x)}{q(x)}
\end{equation}
\begin{equation}
    f(x) = 
        s(x) + 
        \sum_{u = 1}^{v} 
            c_u(x) + 
        \sum_{w = 1}^{z} 
            d_w(x)
\end{equation}
o que nos da finalmente que
\begin{equation}
    \int f(x) dx = 
        \int s(x)dx + 
        \int 
            \left(
                \sum_{u = 1}^{v} c_u(x)
            \right)
        dx
        + 
        \int 
            \left(
                \sum_{w = 1}^{z} d_w(x)
            \right)
        dx
\end{equation}
e como sabemos integral o polinômio \(s(x)\) e cada polinômio \(c_u\) e \(d_w\), temos uma integral simplificada.
\section{Substituição racionalizante}

Se tivermos:
\begin{equation}
    \int \sqrt[\leftroot{-3}\uproot{3}n]{g(x)}\; dx
\end{equation}
Substituímos \( u = \sqrt[\leftroot{-3}\uproot{3}n]{g(x)} \)
\\
Se tivermos \(f(x)\) um quociente entre 2 polinomios com potencias fracionarias de x, então fazemos a substituição 
\begin{equation}
    x = t^k
\end{equation} 
Onde \(k\) é denominador comum entre as frações potencia de \(x\)

\end{document}
