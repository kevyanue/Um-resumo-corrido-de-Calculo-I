\chapter{Como conseguir informações sobre funções}

\section{Máximo e Mínimo}

\begin{definition}[ponto e valor de máximo absoluto]
    Seja \(c \in Df\). Dizemos que \(c\) é (\textit{ponto de}) máximo absoluto ou (\textit{ponto de}) máximo global se:
    
    \begin{equation}
        f(c) \geq f(x), \forall x \in Df
    \end{equation}
    e denominamos o número \(M = f(c)\) como o \textbf{valor} de máximo absoluto.
\end{definition}

\begin{definition}[ponto e valor de mínimo absoluto]
    A definição é análoga para mínimo absoluto e valor de mínimo absoluto
\end{definition}

\begin{definition}[ponto e valor de máximo local]
    Seja \(I \subset Df\) um intervalo aberto. Dizemos que \(c\) é (\textit{ponto de}) máximo local ou (\textit{ponto de}) máximo relativo se:
    
    \begin{equation}
        f(c) \geq f(x), \forall x \in I
    \end{equation}
    e denominamos o número \(M = f(c)\) como o \textbf{valor} de máximo local.
\end{definition}

\begin{definition}[ponto e valor de mínimo local]
    A definição é análoga para mínimo local e valor local
\end{definition}

\section{Ponto crítico}

\begin{definition}
    se \(f'(c) = 0\) ou \(f'(c)\) não existe então \(c\) é um ponto crítico.
\end{definition}

\section{Crescente e Decrescente}

\begin{theorem}
    Suponha \(f\) é diferenciável num intervalo aberto \(I\). Vale que:
    \begin{enumerate}
        \item Se \(f'(x) > 0, \forall x  \in I\) então f é crescente em I
        \item Se \(f'(x) < 0, \forall x  \in I\) então f é decrescente em I
        \item \(f'(x) = 0, \forall x  \in I\) então f é constante em I
    \end{enumerate}
\end{theorem}

\section{Assintotas}

\subsection{Horizontal}

\begin{definition}
    \begin{equation}
        \lim_{x \to \pm \infty} f(x) = L \implies \text{a reta} y = L \text{ é assintota horizontal de \(f\)}
    \end{equation}
\end{definition}

\subsection{Vertical}

\begin{definition}
    \begin{equation}
        \lim_{x \to a^{\pm}} f(x) = \pm \infty \implies \text{a reta} x = a \text{ é assintota vertical de \(f\)}
    \end{equation}
\end{definition}
Podemos achar as assintotas verticais da seguinte maneira
\begin{method}
    Seja \(P\) o conjunto dos pontos onde dado um numero qualquer \( a \in P\), temos que:
    \begin{enumerate}
        \item \(a \notin Df\) ou
        \item \(f\) descontinua em \(a\)
    \end{enumerate}
    Testando os limites laterais em cada um dos \( a \in P\) nos da todas as assintotas verticais.
\end{method}

\subsection{Oblíqua}

\begin{definition}
    \begin{equation}
        \lim_{x \to \pm \infty} [f(x) - (mx + b)] = 0, m \neq 0 \implies 
    \end{equation}
    \centering
    a reta \( y = mx + b\) é assintota oblíqua
\end{definition}
Para encontrar assintota obliqua:
\begin{method}
    Se existir \(m\) e \(b\) tal que:
    \begin{enumerate}
        \item 
            \begin{equation}
                \lim_{x \to \pm \infty}  \frac{f(x)}{x} = m
            \end{equation}
        \item
            \begin{equation}
                \lim_{x \to \pm \infty}  f(x) - mx = b
            \end{equation}
    \end{enumerate}
    Então
    \(y = mx + b\) é uma assintota obliqua
\end{method}

\section{Encontrar máx. ou mín.}

\subsection{intervalo fechado pra valores absoluto}

\begin{method}
    Seja \(f\) uma função contínua em um intervalo \([a,b]\).
    Podemos encontrar o máximo e mínimo absoluto da seguinte forma
    
    \begin{enumerate}
        \item encontre os valores de \(f\) nos pontos criticos em \((a,b)\)
        \item encontre os valores de \(f(a)\) e \(f(b)\) (extremos do intervalo)
        \item O menor valor dos números é o valor absoluto e o menor valor é o mínimo absoluto
        
    \end{enumerate}    
\end{method}



\subsection{Teste da primeira derivada}
\begin{method}
    Seja \(f\) uma função continua e \(c\) um ponto critico. Temos que:
    \begin{enumerate}
        \item se o sinal de \(f'\) mudar de positivo pra negativo em \(c\) \(\implies\) \(c\) é ponto de máximo local
        \item se o sinal de \(f'\) mudar de negativo pra positivo em \(c\) \(\implies\) \(c\) é ponto de máximo local
        \item se o sinal de \(f'\) não mudar, então \(c\) não é ponto de máximo nem de mínimo
    \end{enumerate}    
\end{method}



\subsection{Teste da segunda derivada}

\begin{method}
    Suponha \(f'(c) = 0\), ou seja, as raizes de \(f'(x)\) , e que \(f''(c)\) existe. Então temo que:
    
    \begin{enumerate}
        \item se \(f''(c) > 0 \implies\) f(c) é um valor de mínimo local
        \item se \(f''(c) < 0 \implies\) f(c) é um valor de máximo local
    \end{enumerate}    
\end{method}
O teste falha quando:
\begin{enumerate}
    \item \(\nexists f'(c)\)
    \item \(f''(c) = 0\)
    \item \(\nexists f''(c)\)
\end{enumerate}

\section{Concavidade}

\subsection{Teste de concavidade}

\begin{enumerate}
    \item \(f''(x) > 0,\, \forall x \in I \implies f\) em \(I\) é côncavo para cima
    \item \(f''(x) < 0,\, \forall x \in I \implies f\) em \(I\) é côncavo para baixo
\end{enumerate}

\begin{remark}
    Note que:\\
    \(f''(x) = 0,\, \forall x \in I \\
    \implies f'(x) = c \\
    \implies f(x) = cx + d \\ 
    \implies f\) é reta afim e não tem concavidade
\end{remark}

\subsection{Ponto de inflexão}

\begin{definition}
    Um ponto \(P\) onde a curva \(y = f(x)\) é chamado de \textit{Ponto de Inflexão} se:
    \begin{enumerate}
        \item se \(f\) for contínua em \(P\)
        \item e a curva mudar de concavidade em \(P\)
    \end{enumerate}
\end{definition}


\section{Construção do gráfico}


\begin{method}
    O algoritmo para construção do gráfico:
    \begin{enumerate}
        \item Determine o Domínio \(Df\) de \(f\)
        \item Determine onde existe intersecção com os eixos, os pontos:
            \begin{enumerate}
                \item onde \(f(x) = 0\)
                \item e o ponto \(f(0)\)
            \end{enumerate}
        \item Se existe Simetria.  Se ela é
            \begin{enumerate}
                \item Par: \(f(-x) = f(x), \forall x \in Df \implies\) gráfico simétrico em relação ao eixo \(y\)
                \item Impar: \(f(-x) = -f(x), \forall x \in Df \implies\) gráfico simétrico em relação a reta \(y = -x\)
                \item Periódico: \(f(x + p) = f(x), \forall x \in Df \implies\) o gráfico se repete em cada intervalo e comprimento \(p\) 
            \end{enumerate}
        \item Determinar as assintotas
            \begin{enumerate}
                \item Horizontais
                \item verticais
                \item Oblíquas
            \end{enumerate}
            \begin{remark}
                se existe assintota horizontal \(\implies\) não existe assintota obliquá
            \end{remark}
        \item Intervalos crescentes e decrescentes
        \item Pontos e valores de máximo e mínimo
        \item Concavidade e pontos de inflexão
        \item Esboçe a curva:
            \begin{enumerate}
                \item Marque as assintotas
                \item Marque todos os pontos que achou
                \item Marque os intervalos de concavidade e crescimento/decrescimento
                \item Desenha a curva atendendo aos que esta marcado
            \end{enumerate}
            
    \end{enumerate}
    
    
\end{method}