\chapter{Aproximações}

\section{Aproximação linear}

Aproximamos uma função por sua reta tangente em algum ponto \(x\) por
\begin{equation}
    f(x) \approx L(x) = f'(a).(x-a) + f(a), \quad \text{para x próximos de a}    
\end{equation}
\begin{definition}
    E definimos \(L(x)\) como \textbf{linearização de \(f\) em \(a\)}
\end{definition}

\section{Diferenciais}

\begin{equation}
    \Delta y = f(x + \Delta x) - f(x)
\end{equation}

\begin{definition}[Diferencial dx]
    \begin{equation}
        dx = \Delta x
    \end{equation}
    
\end{definition}

\begin{definition}[Diferencial dy]
    \begin{equation}
        dy = L(x + \Delta x) - L(x)
    \end{equation}
    
\end{definition}

\begin{equation}
    \Delta y = f(x + \Delta x) - f(x) \approx L(x + \Delta x) - L(x)
\end{equation}

\begin{definition}[Erro relativo]
    O erro relativo \(E\) da aproximação é dado por
    \begin{equation}
        E = \frac{\Delta y}{y}    
    \end{equation}
\end{definition}

\section{Polinômio de Taylor}
O polinômio de taylor \(T\) de ordem \(K\) de uma função \(f\) no ponto \(a\) é dada por:
\begin{equation}
    T_K(f(a)) = \sum_{n=0}^{K} \frac{f^{(n)}(a)}{n!}(x - a)
\end{equation}
Onde \(f^{(n)}\) é a \textit{n-ésima} derivada de \(f\). Temos:
\begin{equation}
    T_K(f(a)) = \frac{f^{(0)}(a)}{0!}(x - a) +
           \frac{f^{(1)}(a)}{1!}(x - a) + ... +
           \frac{f^{(K)}(a)}{K!}(x - a)
\end{equation}
Para o valor especifico de \(N= \infty\) e \(f\) infinitamente diferenciável:
\begin{equation}
    T(f(a)) = \sum_{n=0}^{\infty} \frac{f^{(n)}(a)}{n!}(x - a)
\end{equation}
\begin{equation}
    T(f(a)) = \frac{f^{(0)}(a)}{0!}(x - a) +
           \frac{f^{(1)}(a)}{1!}(x - a) +
           \frac{f^{(2)}(a)}{2!}(x - a) + ...
\end{equation}
