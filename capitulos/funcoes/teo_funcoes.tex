\chapter{Teoremas sobre funções}

\section{Teorema do Valor intermediário}

Seja \(f\) uma função contínua no intervalo \(I = [a,b]\), \(N \in I\) um numero qualuer e \(f(a) \neq f(b)\).
Então existe \(c \in I\) tal que f(c) = N
    
\section{Teorema do anulamento (Teorema de Bolzano)}

Seja \(f\) uma função contínua no intervalo \(I = [a,b]\) e \(a \leq 0 \leq b\).
Então existe \(c \in I\) tal que f(c) = 0

\section{Teorema do valor Extremo (Teorema de \textit{Weistrass})}

\begin{theorem}
    Seja f continua em um intervalo fechado \([a,b]\).
    Então existe um \(M\) máximo e \(m\) mínimo contido em \([a,b]\)
\end{theorem}

\section{Teorema de Fermat}

\begin{theorem}
    Se f tiver um máximo ou mínimo local em \(c\), então \(c\) é um ponto crítico.
\end{theorem}

\section{Teorema de Rolle}

\begin{theorem}
    Seja \(f\) uma função tal que:
    \begin{enumerate}
        \item \(f\) contínua em \([a,b]\)
        \item \(f\) diferenciável em \((a,b)\)
        \item \(f(a)\) = \(f(b)\)
    \end{enumerate}
    Então
    \begin{equation}
        \exists c \in (a,b);\quad f'(c) = 0
    \end{equation}
    O que se traduz como: ``se os extremos são iguais então a derivada se anula em algum lugar''
\end{theorem}

\section{Teorema do Valor Médio}
\begin{theorem}
    seja \(f\) uma função. Se:
    \begin{enumerate}
        \item f contínua no intervalo \( [a,b] \)
        \item f diferenciável no intervalo \( (a,b) \)
    \end{enumerate}
    Então:
    
    \begin{equation}
        \exists c \in (a,b);\quad f'(c) = \frac{f(a) - f(b)}{a - b}
    \end{equation}    
\end{theorem}