\chapter{Derivadas}

\begin{equation}
f'(x) 
=
\lim_{x \to 0} \frac{
    f(x + h) - f(x)
}{
    h
}
\end{equation}
Ou
\begin{equation}
f'x 
= 
\lim_{x \to a} \frac{
    f(x) - f(a)
}{
    x - a
}    
\end{equation}


\subsection*{Notações}

temos as seguintes notações para a primeira derivada e as derivadas superiores:
\begin{flalign}
    f'(x), f''(x), f'''(x), ... &&
\end{flalign}
\begin{flalign}
    \frac{d}{dx} f, \frac{d^2}{dx^2} f, \frac{d^3}{dx^3}f , ... &&
\end{flalign}
\begin{flalign}
    D^1 f(x), D^2 f(x), D^3 f(x), ... &&
\end{flalign}
\begin{flalign}
    f^{(1)}, f^{(2)}, f^{(3)}, ... &&
\end{flalign}
OBS: Note que a \(n\) derivada \(f^{\textbf{(n)}}(x)\) é diferente da \(n\) composta \(f^{\textbf{n}}(x)\) apesar das notações serem parecidas. 

\section{Diferenciabilidade e continuidade}

\begin{definition}
    \(f\) diferenciável em \(a\) se \(f'(a)\) existe    
\end{definition}

\begin{theorem}
    \begin{equation}
        \text{f diferenciável} \implies \text{f continua}
    \end{equation}
    
    Não vale a volta.
\end{theorem}

\begin{theorem}
    \begin{equation}
        \text{f descontinua em a} \implies \text{f não-diferenciável em a}  
    \end{equation}
    
    Não vale a volta.
\end{theorem}

\section{reta rangente}

A reta \(r\) tangente a um ponto \((a, f(a))\)
é da por

\begin{equation}
y = f'(a)(x - a) + f(a)    
\end{equation}


\section{Regras de derivação}

\textbf{multiplicação por escalar}

\begin{flalign}
    \frac{d}{dx} \left( cf(x) \right) = c \frac{d}{dx} f(x)    &&
\end{flalign}
\\
\textbf{Soma}
\\
\begin{flalign}
    \frac{d}{dx} (f + g)(x) =
    \frac{d}{dx} f(x) + \frac{d}{dx} g(x) &&
\end{flalign}
\\
\textbf{Multiplicação}
\\
\begin{flalign}
    \frac{d}{dx} (fg)(x) = f'(x)g(x) + f(x)g'(x) &&
\end{flalign}
\\
\textbf{Divisão}
\\
\begin{flalign}
    \frac{d}{dx} \left( \frac{f}{g} \right) (x)
    = \frac{f'(x)g(x) - f(x)g'(x)}{g(x)^2} &&
\end{flalign}
\\
\textbf{Composta}
\\
\begin{flalign}
    \frac{d}{dx} (f \circ g)(x) = (f' \circ g)(x) g'(x) &&
\end{flalign}
\\
\textbf{Inversa}
\\
\begin{flalign}
    \frac{d}{dx} f^{-1}(y) = \frac{1}{(f' \circ f^{-1})(y)} &&
\end{flalign}

\section{Tabela de derivadas}

\subsection{Derivadas simples}

\begin{flalign}
    \frac{d}{dx} c = 0 &&
\end{flalign}
\begin{flalign}
    \frac{d}{dx} x = 1 &&
\end{flalign}
\begin{flalign}
    \frac{d}{dx} x^{n} = nx^{n-1} &&
\end{flalign}
\begin{flalign}
    \frac{d}{dx} \frac{1}{x^n} =
    \frac{d}{dx} (x^{-c}) = -\frac{c}{x^{c+1}} &&
\end{flalign}
\begin{flalign}
    \frac{d}{dx} \sqrt{x} = x^{\frac{1}{2}} = -\frac{1}{2} x^{-\frac{1}{2}} = -\frac{1}{2\sqrt{x}} &&
\end{flalign}
\begin{flalign}
    &\frac{d}{dx} e^x = e^x &&
\end{flalign}
\begin{flalign}
    &\frac{d}{dx} e^{cx} = ce^{cx} &&
\end{flalign}
\begin{flalign}
    &\frac{d}{dx} \log_a x = \frac{1}{ln(a)}. \frac{1}{x} &&
\end{flalign}
\begin{flalign}
    &\frac{d}{dx} \ln{x} = \frac{1}{x} &&
\end{flalign}
\begin{flalign}
    &\frac{d}{dx} a^x = a^x ln(a) &&
\end{flalign}

\subsection{Derivadas de trigonométricas}

\textbf{Macete}

se tem Derivada \(cos\) aparece sinal de \(-\) e derivadas do \(sen\) não tem sinal adicional.

e segue essa sequencia

\begin{equation}
      sen(x) \stackrel{D}{\longrightarrow} 
      cos(x) \stackrel{D}{\longrightarrow} 
    - sen(x) \stackrel{D}{\longrightarrow} 
    - cos(x) \stackrel{D}{\longrightarrow}
      sen(x) \stackrel{D}{\to} ...
\end{equation}

\begin{flalign}
    \frac{d}{dx} sen(x) = cos(x) &&
\end{flalign}

\begin{flalign}
    \frac{d}{dx} cos(x) = -sen(x) &&
\end{flalign}

\begin{flalign}
    \frac{d}{dx} -sen(x) = -cos(x) &&
\end{flalign}

\begin{flalign}
    \frac{d}{dx} -cos(x) = sen(x) &&
\end{flalign}

\begin{flalign}
    \frac{d}{dx} tan(x) = sec^2(x) &&
\end{flalign}

\begin{flalign}
    \frac{d}{dx} sec(x) = tan(x)sec(x) &&
\end{flalign}

\begin{flalign}
    \frac{d}{dx} cotan(x) = -cossec^2(x) &&
\end{flalign}

\begin{flalign}
    \frac{d}{dx} cosec(x) = -cotan(x)cosec(x) &&
\end{flalign}




\subsection{Derivadas de trigonométricas inversas}

Passo a passo geral, onde \(trig\) é uma função trigonométrica qualquer e \(arctrig\) é sua inversa
\begin{enumerate}
    \item estabeleça o domínio de \(trig\)
    \item note que \(y = trig(x) \iff arctrig(y) = x\)
    \item Diferencie implicitamente
    \item isole o \(y'\) como usualmente se faz e terá a derivada da inversa trigonométrica
\end{enumerate}

\begin{flalign}
    \frac{d}{dx} arcsen(x) = \frac{1}{\sqrt{1 - x^2}} &&
\end{flalign}

\begin{flalign}
    \frac{d}{dx} arccos(x) = - \frac{1}{\sqrt{1 - x^2}} &&
\end{flalign}

\begin{flalign}
    \frac{d}{dx} arctan(x) = \frac{1}{1 + x^2} &&
\end{flalign}

\begin{flalign}
    \frac{d}{dx} arccotan(x) = - \frac{1}{1 + x^2} &&
\end{flalign}

\begin{flalign}
    \frac{d}{dx} arcsec(x) = \frac{1}{x \sqrt{x^2 - 1}} &&
\end{flalign}

\begin{flalign}
    \frac{d}{dx} arccosec(x) = - \frac{1}{x \sqrt{x^2 - 1}} &&
\end{flalign}

\subsection{Derivadas de potencias}

\begin{flalign}
    \frac{d}{dx} x^x = x^x(ln(x) + 1),\quad para\; x>0 &&
\end{flalign}

\begin{flalign}
    \frac{d}{dx} g(x)^a = a.g(x)^{a-1}.g'(x) &&
\end{flalign}

\begin{flalign} 
    \frac{d}{dx} a^{g(x)} = ln(a).a^{g(x)}.g'(x) &&
\end{flalign}

\begin{flalign}
    \frac{d}{dx} f(x)^{g(x)} = \frac{d}{dx}  \, e^{g(x).ln(f(x))} &&
\end{flalign}

\begin{flalign}
     = f(x)^{g(x)} . g'(x) . ln (f(x)) + f(x)^{g(x) - 1}.g(x).f'(x) &&
\end{flalign}