\chapter{Trigonométricas hiperbólicas}
\section{Trigonométricas hiperbólicas}

\begin{flalign}
     senh(x) = \frac{e^x - e^{-x}}{2} &&
\end{flalign}

\begin{flalign}
     cosh(x) = \frac{e^x + e^{-x}}{2} &&
\end{flalign}
As demais funções trigonométricas são definidas como usualmente se faz, mas com \(senh\) e \(cosh\) no lugar de \(sen\) e \(cos\)

\section{Identidades trigonométricas hiperbólicas}

\begin{flalign}
    senh(x) = -senh(x), \quad \text{(senh é impar)} &&
\end{flalign}
\begin{flalign}
    cosh(-x) = cosh(x), \quad \text{(cosh é par)} &&
\end{flalign}
\begin{flalign}
    cosh^2(x) -senh^2(x) = 1 &&
\end{flalign}
\begin{flalign}
    1 - tanh^2(x) = sech^2(x) &&
\end{flalign}
\begin{flalign}
    senh(a + b) = senh(a)cosh(b) + cosh(a)senh(b) &&
\end{flalign}
\begin{flalign}
    cosh(a + b) = cosh(a)cosh(b) + senh(a)senh(b) &&
\end{flalign}

\section{Derivadas de trigonométricas hiperbólicas}

\begin{flalign}
    \frac{d}{dx} senh(x) = cosh(x) &&
\end{flalign}
\begin{flalign}
    \frac{d}{dx} cosh(x) = senh(x) &&
\end{flalign}
\begin{flalign}
    \frac{d}{dx} tanh(x) = sech^2(x) &&
\end{flalign}
\begin{flalign}
    \frac{d}{dx} sech(x) = - tangh(x) sech^2(x) &&
\end{flalign}
\begin{flalign}
    \frac{d}{dx} cotangh(x) = - cossech^2(x) &&
\end{flalign}
\begin{flalign}
    \frac{d}{dx} cossech(x) = - cotanh(x) cossec(x) &&
\end{flalign}

\section{Derivadas de inversa trigonométricas hiperbólicas}

\begin{flalign}
    \frac{d}{dx} arcsenh(x) = \frac{1}{\sqrt{1 + x^2}} &&
\end{flalign}

\begin{flalign}
    \frac{d}{dx} arccosh(x) = \frac{1}{\sqrt{-1 + x^2}} &&
\end{flalign}

\begin{flalign}
    \frac{d}{dx} arccosh(x) = \frac{1}{\sqrt{-1 + x^2}} &&
\end{flalign}

\begin{flalign}
    \frac{d}{dx} arctanh(x) = \frac{1}{1 - x^2} &&
\end{flalign}
